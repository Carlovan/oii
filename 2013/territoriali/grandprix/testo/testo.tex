\documentclass[a4paper,11pt]{article}

\usepackage[utf8x]{inputenc}
\SetUnicodeOption{mathletters}
\SetUnicodeOption{autogenerated}

\usepackage[italian]{babel}
\usepackage{booktabs}
\usepackage{mathpazo}
\usepackage{graphicx}
\usepackage[left=2cm, right=2cm, bottom=3cm]{geometry}
\frenchspacing

\begin{document}
\noindent {\Large Selezioni Territoriali 2012}
\vspace{0.5cm}

\noindent {\Huge Grand Prix (\texttt{grandprix})}


\vspace{0.5cm}
\noindent {\Large Difficoltà D = 1.}

\section*{Descrizione del problema}
  
State assistendo a un Gran Premio di Formula 1. Prima dell'inizio, il
tabellone riporta la griglia di partenza, ovvero l'ordine in cui le
vetture partiranno dalla linea del traguardo. Non appena inizia il gran
premio, per ogni sorpasso, il tabellone scrive due numeri: quello della
vettura che ha effettuato il sorpasso, e quello della vettura che è
stata superata. Il vostro compito è di scrivere un programma che,
ricevuti in ingresso l'ordine di partenza e la lista dei sorpassi,
calcoli chi ha vinto il gran premio.  Per esempio, considerate il
seguente gran premio, con 3 macchine e 4 sorpassi. L'ordine iniziale di
partenza è stato: la vettura numero 2, poi la vettura numero 1 e infine
la vettura numero 3. I sorpassi sono stati, nell'ordine:
    
\begin{itemize}
  \item la numero 3 ha superato la numero 1;
  \item la numero 3 ha superato la numero 2;
  \item la numero 1 ha superato la numero 2;
  \item la numero 2 ha superato la numero 1;
\end{itemize}

In questo caso, è facile vedere che la vettura numero 3 ha vinto il gran
premio. Come si può notare dall'esempio, i sorpassi avvengono sempre tra
due vetture consecutive.
    
\section*{Dati di input}
  
Il file di input è costituito da $1+N+M$ righe di testo.  La prima riga
contiene due interi positivi separati da uno spazio: $N$ che è il numero
di vetture e $M$ che è il numero di sorpassi. Le successive $N$ righe
contengono l'ordine di partenza: per ogni riga c'è un numero intero $K$
che rappresenta una vettura, con $1 ≤ K ≤ N$. La vettura che parte in
$i$-esima posizione nell'ordine di partenza si trova quindi nella riga
$(i+1)$ del file.  Le restanti M righe contengono tutti i sorpassi,
nell'ordine in cui sono avvenuti, uno in ogni riga. Ogni riga contiene
due interi separati da uno spazio: $A$, ovvero il numero della vettura
che ha effettuato il sorpasso, e $B$, ovvero il numero della vettura che
ha subito il sorpasso. 

\section*{Dati di output}
  
Il file di output deve contenere un solo intero: il numero della vettura
che ha vinto il gran premio.
    
\section*{Assunzioni}
\begin{itemize}
  \item $2 ≤ N ≤ 30$
  \item $1 ≤ M ≤ 100$
\end{itemize}

\section*{Esempi di input/output}
    \noindent
    \begin{tabular}{p{11cm}|p{5cm}}
    \toprule
    \textbf{File \texttt{input.txt}}
    & \textbf{File \texttt{output.txt}}
    \\
    \midrule
    \scriptsize
    \begin{verbatim}
3 4
2
1
3
3 1
3 2
1 2
2 1
      \end{verbatim}
    &
    \scriptsize
    \begin{verbatim}
3
      \end{verbatim}
    \\
    \bottomrule
    \end{tabular}
  
    \noindent
    \begin{tabular}{p{11cm}|p{5cm}}
    \toprule
    \textbf{File \texttt{input.txt}}
    & \textbf{File \texttt{output.txt}}
    \\
    \midrule
    \scriptsize
    \begin{verbatim}
3 6
3
1
2
2 1
1 2
1 3
2 3
2 1
3 1
      \end{verbatim}
    &
    \scriptsize
    \begin{verbatim}
2
      \end{verbatim}
    \\
    \bottomrule
    \end{tabular}

\end{document}
