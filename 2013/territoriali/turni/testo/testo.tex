\documentclass[a4paper,11pt]{article}

\usepackage[utf8x]{inputenc}
\SetUnicodeOption{mathletters}
\SetUnicodeOption{autogenerated}

\usepackage[italian]{babel}
\usepackage{booktabs}
\usepackage{mathpazo}
\usepackage{graphicx}
\usepackage[left=2cm, right=2cm, bottom=3cm]{geometry}
\frenchspacing

\begin{document}
\noindent {\Large Selezioni Territoriali 2012}
\vspace{0.5cm}

\noindent {\Huge Turni di guardia (\texttt{turni})}


\vspace{0.5cm}
\noindent {\Large Difficoltà D = 2.}

\section*{Descrizione del problema}
  
La Banda Bassotti è stata rimessa in libertà. Zio Paperone, in partenza
per un viaggio di $K$ giorni, ha la necessità di far sorvegliare il
deposito: quindi ha bisogno che sia sempre presente almeno una persona.
Per risparmiare, decide di chiedere la disponibilità di amici e parenti,
e ognuno di questi fornisce un intervallo di giorni in cui è disponibile
per la sorveglianza. Paperone però sa che dovrà fare un regalo a ognuna
delle persone che userà, e volendo risparmiare al massimo deve
coinvolgere il minimo numero di persone, senza lasciare mai il deposito
scoperto. In questo modo riuscirà a risparmiare sui regali.
    
Per esempio, supponiamo che il viaggio di Zio Paperone sia di $K=8$
giorni, con partenza il giorno $0$ e ritorno il giorno $K-1=7$  e che le
disponibilità siano le seguenti (per ogni nome, tra parentesi si
indicano il giorno iniziale e il giorno finale della disponibilità).

\begin{itemize}
  \item Paperino $(3,5)$
  \item Paperoga $(0,2)$
  \item Battista $(1,3)$
  \item Gastone $(5,6)$
  \item Archimede $(4,7)$
\end{itemize}

In questo caso, a Zio Paperone basta coinvolgere Paperoga, Paperino e
Archimede per assicurarsi che il deposito sia sempre sorvegliato, e se
la cava con tre regali.

Sapendo il numero di giorni di viaggio, e le disponibilità di ognuno, il
vostro compito è quello di aiutare Zio Paperone a calcolare il minimo
numero di persone che servono ad assicurare una sorveglianza continua al
deposito. 


\section*{Dati di input}
  
Il file di input è costituito da $2+N$ righe. La prima riga contiene un
intero positivo $K$, ovvero il numero di giorni del viaggio. La seconda
riga contiene un intero positivo $N$, il numero di persone che hanno
dato la disponibilità a Zio Paperone. Le restanti $N$ righe contengono
una coppia di interi $A$ e $B$ per ognuna delle $N$ persone: questa
coppia di interi rappresenta l'inizio e la fine della disponibilità
della i-esima persona.  
    
\section*{Dati di output}
  
Il file di output deve contenere un solo intero positivo $R$, che è il
numero minimo di persone necessarie ad assicurare una sorveglianza
continua al deposito. 
    
\section*{Assunzioni}
\begin{itemize}
  \item $1 ≤ K, N ≤ 50$
  \item Per ognuna delle $N$ righe, si ha $0 ≤ A ≤ B ≤ K-1$
  \item  Esiste sempre almeno una soluzione in ognuno dei casi di input.
\end{itemize}

\section*{Esempi di input/output}
    \noindent
    \begin{tabular}{p{11cm}|p{5cm}}
    \toprule
    \textbf{File \texttt{input.txt}}
    & \textbf{File \texttt{output.txt}}
    \\
    \midrule
    \scriptsize
    \begin{verbatim}
8
5
3 5
0 2
1 3
5 6
4 7
      \end{verbatim}
    &
    \scriptsize
    \begin{verbatim}
3
      \end{verbatim}
    \\
    \bottomrule
    \end{tabular}
  
    \noindent
    \begin{tabular}{p{11cm}|p{5cm}}
    \toprule
    \textbf{File \texttt{input.txt}}
    & \textbf{File \texttt{output.txt}}
    \\
    \midrule
    \scriptsize
    \begin{verbatim}
10
6
2 5
0 2
1 3
5 6
4 7
7 9
      \end{verbatim}
    &
    \scriptsize
    \begin{verbatim}
4
      \end{verbatim}
    \\
    \bottomrule
    \end{tabular}

\end{document}
