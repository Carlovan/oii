
\documentclass[a4paper,11pt]{article}

\usepackage[utf8x]{inputenc}
\SetUnicodeOption{mathletters}
\SetUnicodeOption{autogenerated}

\usepackage[italian]{babel}
\usepackage{booktabs}
\usepackage{mathpazo}
\usepackage{graphicx}
\usepackage[left=2cm, right=2cm, bottom=3cm]{geometry}
\frenchspacing

\begin{document}
\noindent {\Large Selezioni territoriali 2009}
\vspace{0.5cm}

\noindent {\Huge Depurazione dell'acqua (\texttt{depura})}


\vspace{0.5cm}
\noindent {\Large Difficoltà D = 2.}

\section*{Descrizione del problema}
   
Bisogna realizzare un procedimento chimico per la depurazione
dell'acqua, avendo a disposizione un certo numero di sostanze,
numerate da 1 in avanti. Per un'efficace depurazione, è
necessario inserire nell'acqua la sostanza chimica purificante
numero 1, tenendo presente che nell'acqua sono già
presenti $K$ sostanze chimiche.

Per quanto riguarda il procedimento adottato, valgono $R$
precise regole per poter inserire le sostanze chimiche
nell'acqua. Tali regole prevedono che una certa sostanza A possa
essere inserita solo se nell'acqua sono già presenti un dato insieme
di sostanze, ad esempio, A1, A2,..., An (dove Ai ≠ A per 1 ≤ i
≤ n).  In tal caso, scriviamo tale regola di inserimento nel
seguente modo

A :-- A1, A2,..., An\\
e diciamo che A compare nella parte sinistra della regola. Al fine di
un corretto inserimento delle sostanze, valgono le seguenti
osservazioni:


\begin{itemize}
  
    \item l'eventuale presenza di ulteriori sostanze non inibisce
l'applicabilità della regola suddetta;
    \item se A compare nella parte sinistra di una regola, allora non
  può comparire nella parte sinistra di altre regole e non
  può essere una delle $K$ sostanze già
  presenti nell'acqua;
    \item qualora una sostanza sia priva di regole (ossia non compaia mai nella
  parte sinistra di una qualche regola) e non sia già presente
  nell’acqua, tale sostanza non può essere inserita;
    \item non è necessario usare tutte le regole e/o tutte le
  sostanze a disposizione.
\end{itemize}


Per esempio, ipotizzando che  le sostanze 2 e 3 siano già
presenti nell'acqua ($K$=2) e che valgano le seguenti regole
($R$=4):\\
4 :-- 2\\
5 :-- 2, 3\\
7 :-- 2, 4\\
1 :-- 3, 7, 4\\
possiamo inserire la sostanza 4 perché la sostanza 2 è
già presente (prima regola); in seguito, possiamo inserire
anche la sostanza 7 perché le sostanze 2 e 4 sono presenti
nell'acqua (terza regola); a questo punto, possiamo aggiungere la
sostanza 1 perché le sostanze 3, 7 e 4 sono presenti (ultima
regola). Quindi abbiamo inserito un totale di $S$=3 sostanze, ossia
4, 7 e 1 (oltre alle $K$=2 già presenti), per
purificare l'acqua.

Scrivere un programma che calcoli il numero minimo $S$ di
sostanze da inserire per purificare l'acqua, conoscendo
le $K$ sostanze già presenti nell'acqua e
le $R$ regole di inserimento.  Tale numero
sarà $S$ = 0 se la sostanza 1 è già
presente nell'acqua; sarà $S$ = 1 se la sostanza 1
può essere inserita direttamente e non è già
presente; in generale, sarà $S$ = $m$ se
è necessario inserire $m$-1 sostanze prima di poter
inserire la sostanza 1. Nel caso in cui non sia possibile purificare
l'acqua, bisogna restituire il valore $S$ = -1.


\section*{Dati di input}
  
Il file \texttt{input.txt} è composto da $K$+$R$+1 righe.

La prima riga contiene due interi positivi separati da uno spazio,
rispettivamente il numero $K$ delle sostanze chimiche già
presenti nell'acqua e il numero $R$ di regole di inserimento.

La successive $K$ righe contengono le $K$ sostanze
già presenti nell'acqua, dove ogni riga è composta da un
solo intero positivo che rappresenta una di tali sostanze.

Le ultime $R$ righe rappresentano le $R$ regole, al
massimo una regola per ciascuna sostanza non presente nell'acqua.
Ciascuna riga è composta da $n$+2 interi positivi A, n, A1,
A2,..., An separati da uno spazio (dove Ai ≠ A per 1 ≤ i ≤
n), i quali rappresentano la regola A :-- A1, A2,..., An.


\section*{Dati di output}
  
Il file \texttt{output.txt} è composto da una sola riga
contenente un intero $S$, il minimo numero di sostanze
inserite (oltre alle $K$ già presenti) per purificare l'acqua
secondo le regole descritte sopra.

  \section*{Assunzioni}
  \begin{itemize}
  
    \item  1 ≤ $K, R$ ≤ 1000
    \item Il numero di sostanze chimiche a disposizione è al massimo 2000.
    \item I casi di prova non contengono mai situazioni cicliche: in tal
modo, non accade mai che una sostanza A possa essere inserita solo
se A stessa è già presente nell'acqua.
  \end{itemize}

\section*{Esempi di input/output}

  
    \noindent
    \begin{tabular}{p{11cm}|p{5cm}}
    \toprule
    \textbf{File \texttt{input.txt}}
    & \textbf{File \texttt{output.txt}}
    \\
    \midrule
    \scriptsize
    \begin{verbatim}
2 4
2
3
4 1 2
5 2 2 3
7 2 2 4
1 3 3 7 4
\end{verbatim}
    &
    \scriptsize
    \begin{verbatim}
3
\end{verbatim}
    \\
    \bottomrule
    \end{tabular}
  
    \noindent
    \begin{tabular}{p{11cm}|p{5cm}}
    \toprule
    \textbf{File \texttt{input.txt}}
    & \textbf{File \texttt{output.txt}}
    \\
    \midrule
    \scriptsize
    \begin{verbatim}
4 2
6
2
8
3
5 2 2 6
1 2 3 6
\end{verbatim}
    &
    \scriptsize
    \begin{verbatim}
1
\end{verbatim}
    \\
    \bottomrule
    \end{tabular}
  
    \noindent
    \begin{tabular}{p{11cm}|p{5cm}}
    \toprule
    \textbf{File \texttt{input.txt}}
    & \textbf{File \texttt{output.txt}}
    \\
    \midrule
    \scriptsize
    \begin{verbatim}
2 3
1
3
4 1 2
5 1 3
6 2 2 4
\end{verbatim}
    &
    \scriptsize
    \begin{verbatim}
0
\end{verbatim}
    \\
    \bottomrule
    \end{tabular}
  
    \noindent
    \begin{tabular}{p{11cm}|p{5cm}}
    \toprule
    \textbf{File \texttt{input.txt}}
    & \textbf{File \texttt{output.txt}}
    \\
    \midrule
    \scriptsize
    \begin{verbatim}
3 4
2
4
8
5 2 2 4
7 2 4 3
6 2 5 7
1 3 5 2 6
\end{verbatim}
    &
    \scriptsize
    \begin{verbatim}
-1
\end{verbatim}
    \\
    \bottomrule
    \end{tabular}
  
\section*{Nota/e}
\begin{itemize}
  
    \item Il numero di sostanze chimiche a disposizione può essere
  semplicemente dedotto guardando il massimo intero contenuto nel
  file \texttt{input.txt}.
    \item Il numero $R$ di regole di inserimento può essere
  inferiore al numero di sostanze a disposizione e non presenti nell'acqua.
    \item Un programma che restituisce sempre lo stesso valore,
indipendentemente dai dati in \texttt{input.txt}, non totalizza
alcun punteggio in aggiunta a quello ottenuto per la sua
compilazione.
\end{itemize}



\end{document}
