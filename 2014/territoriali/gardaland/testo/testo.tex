% Template per generare

\documentclass[a4paper,11pt]{article}
\usepackage{lmodern}
\renewcommand*\familydefault{\sfdefault}
\usepackage{sfmath}
\usepackage[utf8]{inputenc}
\usepackage[T1]{fontenc}
\usepackage[italian]{babel}
\usepackage{indentfirst}
\usepackage{graphicx}
\usepackage{tikz}
\newcommand*\circled[1]{\tikz[baseline=(char.base)]{
		\node[shape=circle,draw,inner sep=2pt] (char) {#1};}}
\usepackage{enumitem}
% \usepackage[group-separator={\,}]{siunitx}
\usepackage[left=2cm, right=2cm, bottom=3cm]{geometry}
\frenchspacing

\newcommand{\num}[1]{#1}

% Macro varie...
\newcommand{\file}[1]{\texttt{#1}}
\renewcommand{\arraystretch}{1.3}
\newcommand{\esempio}[2]{
\noindent\begin{minipage}{\textwidth}
\begin{tabular}{|p{11cm}|p{5cm}|}
	\hline
	\textbf{File \file{input.txt}} & \textbf{File \file{output.txt}}\\
	\hline
	\tt \small #1 &
	\tt \small #2 \\
	\hline
\end{tabular}
\end{minipage}
}

% Dati del task
\newcommand{\gara}{Olimpiadi Italiane di Informatica - Selezioni Territoriali 2013}
\newcommand{\nome}{Gita a Gardaland}
\newcommand{\nomebreve}{gardaland}

\begin{document}


% Intestazione
\noindent{\Large \gara}
\vspace{0.5cm}

\noindent{\Huge \textbf \nome~(\texttt{\nomebreve})}
\vspace{0.2cm}\\
\noindent{\large \textsc{Difficoltà D=1}}

% Descrizione del task
\section*{Descrizione del problema}
Nel 2012 le Olimpiadi Internazionali di Informatica (IOI) si sono
svolte, per la prima volta, in Italia, a Sirmione.  Come da
tradizione, nella giornata tra le due gare i concorrenti sono andati a
divertirsi in un parco giochi, in questo caso, Gardaland.  La mattina
di quel giorno decine di pullman hanno prelevato i quattro ragazzi che
costituiscono la squadra olimpica di ciascuna nazione dal Garda
Village, dove erano stati alloggiati, e li hanno portati a Gardaland.

Come sempre negli spostamenti, le varie nazioni erano state ripartite
a blocco unico tra i pullman, ossia tutti gli atleti di una stessa
nazione trovavano posto su uno stesso pullman.  Per esempio, sul
pullman dell’Italia viaggiavano anche Giappone, Israele e Irlanda.  Al
ritorno però, come sempre succede alle IOI, dopo una giornata in un
parco giochi i ragazzi hanno fatto amicizia tra di loro, e al momento
di tornare sui pullman sono saliti alla rinfusa.

Grazie al lavoro delle guide, per ogni pullman è stata stilata una
lista contenente, per ogni nazione, il numero di ragazzi a bordo.  Il
vostro compito è quello di aiutare Monica, responsabile
dell’organizzazione, a capire se i pullman possono partire, ovvero se
tutti i quattro ragazzi di ogni nazione che sono arrivati a Gardaland
sono saliti sui pullman.  In caso contrario, dovete segnalare a Monica
in quanti mancano all’appello, divisi per nazioni.

\section*{Dati di input}
Il file \verb'input.txt' è composto da $1+N+L$ righe.  La prima riga
contiene due interi positivi separati da uno spazio: il numero $N$
delle nazioni e il numero $L$ di righe contenenti informazioni su chi
è attualmente già salito sui pullman.  (Ciascuna nazione verrà qui
rappresentata con un intero compreso tra $0$ e $N-1$).  Ognuna delle
successive $N$ righe contiene un intero positivo: nella riga $i+1$
(con $i \ge 1$) troviamo il numero totale di ragazzi della nazione
$i-1$.  Ciascuna delle rimanenti $L$ righe contiene due interi
positivi: un intero compreso tra $0$ e $N-1$ che rappresenta la
nazione, e un intero positivo che specifica quanti ragazzi di quella
nazione sono su un certo pullman.  Ovviamente una stessa nazione può
comparire diverse volte nelle $L$ righe, e più precisamente compare su
tante righe quanti sono i pullman ospitanti atleti di quella nazione.

\section*{Dati di output}
Il file \verb'output.txt' è composto da una sola riga contenente
l'intero $0$ (zero) se non manca alcun ragazzo.  Altrimenti, il file
contiene $1+C$ righe: la prima riga contiene un intero $C$, ovvero il
numero di nazioni che hanno ragazzi ancora a Gardaland.  Le restanti
$C$ righe contengono due interi: l'identificativo della nazione e il
numero di ragazzi di quella nazione che non sono ancora saliti su
alcun pullman.  E’ necessario stampare le nazioni nell'ordine in cui
sono state lette, ovvero in ordine crescente in base
all'identificativo.

% Assunzioni
\section*{Assunzioni}
\begin{itemize}[nolistsep, noitemsep]
  \item $2 \le N \le 100$
  \item $N \le L \le 1000$
  \item Contrariamente alle olimpiadi di informatica reali, dove
    gareggiano (massimo) $4$ ragazzi per ogni nazione, nei casi di
    input si assume che ogni nazione abbia al massimo 100 ragazzi, e
    almeno $1$ ragazzo. Quindi, indicando con $R_i$ il numero di
    ragazzi della $i$-esima nazione, vale sempre $1 \le R_i \le 100$.
\end{itemize}

% Esempi
\section*{Esempio di input/output}
\esempio{
3 5

4

4

3

0 2

1 3

0 1

2 2

1 1
}{
2

0 1

2 1
}

\esempio{
3 6

4

4

4

0 2

1 3

2 1

0 2

2 3

1 1
}{0}

\section*{Nota}
Un programma che restituisce sempre lo stesso valore, indipendentemente dai dati in \texttt{input.txt}, non totalizza alcun punteggio.

\end{document}
