\documentclass[a4paper,11pt]{article}

\usepackage[utf8x]{inputenc}
\SetUnicodeOption{mathletters}
\SetUnicodeOption{autogenerated}

\usepackage[italian]{babel}
\usepackage{booktabs}
\usepackage{mathpazo}
\usepackage{graphicx}
\usepackage[left=2cm, right=2cm, bottom=3cm]{geometry}
\frenchspacing

\begin{document}
\noindent {\Large Regionali 2004/2005}
\vspace{0.5cm}

\noindent {\Huge Sunnydale (\texttt{sunny})}


\vspace{0.5cm}
\noindent {\Large Difficoltà D = 2.}

\section*{Descrizione del problema}
  Sunnydale è una città che - per
ragioni storiche e ambientali - ospita un elevatissimo numero di
vampiri.Per ragioni cutanee i vampiri non possono sopportare la luce solare e,
storicamente, hanno sempre avuto enormi difficoltà a
viaggiare col sole alto nel cielo; l'attraversamento delle
gallerie sotterranee di Sunnydale è sempre stato
il mezzo preferito dai vampiri per muoversi nella città.I continui crolli delle gallerie hanno creato dei fori nei soffitti,
rendendone alcune troppo luminose per un attraversamento tranquillo e sereno.Harmony, una ragazza-vampiro, passeggia per le
gallerie di Sunnydale quando il suo amico Spike le telefona per invitarla a casa sua.Purtroppo ella si muove per le gallerie sotterranee
secondo una regola tanto semplice quanto tassativa: ad ogni svincolo
sceglie sempre e comunque la galleria meno luminosa per paura di rovinare la propria pelle.Sapendo che non esistono due gallerie egualmente luminose, bisogna determinare
se Harmony possa raggiungere la casa sotterranea di Spike e, in caso
affermativo, quante gallerie le sono necessarie per arrivare.

\section*{Dati di input}
  La prima riga del file \texttt{input.txt} è composta da
quattro numeri interi $N$, $M$, $H$ e
$S$: il primo rappresenta il numero degli svincoli
(numerati da 1 a $N$), il secondo rappresenta il numero delle
gallerie, il terzo rappresenta l'indice dello svincolo in
cui si trova Harmony quando riceve la telefonata; il quarto, infine, rappresenta l'indice
dello svincolo della casa di Spike.Ognuna delle successive $M$ righe descrive una galleria e
contiene tre numeri interi $A$, $B$ e
$L$ separati da uno spazio: i primi due rappresentano gli svincoli collegati dalla
galleria mentre il terzo rappresenta il suo grado di luminosità.

\section*{Dati di output}
  Il file \texttt{output.txt} dovrà contenere un
unico numero intero: -1 se Harmony non riuscirà a raggiungere
Spike; altrimenti, il numero di gallerie che ella percorrerà prima di
raggiungerlo.
  \section*{Assunzioni}
  \begin{itemize}
  
    \item $2 ≤ N ≤ 50000$
    \item $1 ≤ M ≤ 50000$
    \item Non esistono due gallerie con la stessa luminosità $L$.
    \item Per ogni galleria, $1 ≤ L ≤ M$.
    \item $1 ≤ H, S ≤ N$
  \end{itemize}

\section*{Esempi di input/output}

  
    \noindent
    \begin{tabular}{p{11cm}|p{5cm}}
    \toprule
    \textbf{File \texttt{input.txt}}
    & \textbf{File \texttt{output.txt}}
    \\
    \midrule
    \scriptsize
    \begin{verbatim}5 6 1 5
1 2 5
2 3 1
3 4 3
4 5 2
5 1 6
1 4 4\end{verbatim}
    &
    \scriptsize
    \begin{verbatim}2\end{verbatim}
    \\
    \bottomrule
    \end{tabular}
  
    \noindent
    \begin{tabular}{p{11cm}|p{5cm}}
    \toprule
    \textbf{File \texttt{input.txt}}
    & \textbf{File \texttt{output.txt}}
    \\
    \midrule
    \scriptsize
    \begin{verbatim}3 2 2 1
3 1 2
2 3 1\end{verbatim}
    &
    \scriptsize
    \begin{verbatim}-1\end{verbatim}
    \\
    \bottomrule
    \end{tabular}
  
\section*{Nota/e}
\begin{itemize}
  
    \item I nomi Sunnydale, Harmony e Spike sono proprietà di Joss Whedon e Mutant Enemy Inc.
\end{itemize}

\end{document}
