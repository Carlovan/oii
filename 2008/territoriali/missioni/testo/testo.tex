
\documentclass[a4paper,11pt]{article}

\usepackage[utf8x]{inputenc}
\SetUnicodeOption{mathletters}
\SetUnicodeOption{autogenerated}

\usepackage[italian]{babel}
\usepackage{booktabs}
\usepackage{mathpazo}
\usepackage{graphicx}
\usepackage[left=2cm, right=2cm, bottom=3cm]{geometry}
\frenchspacing

\begin{document}
\noindent {\Large Selezioni regionali 2008}
\vspace{0.5cm}

\noindent {\Huge Missioni segrete (\texttt{missioni})}


\vspace{0.5cm}
\noindent {\Large Difficoltà D = 2.}

\section*{Descrizione del problema}
   
Il Commissario Basettoni ha presentato a Topolino le missioni che egli
dovrà svolgere segretamente nel corso dell'anno. Per ogni
missione, oltre al luogo da raggiungere, Basettoni ne indica la durata
in giorni e la data massima entro cui deve essere completata. In altri
termini, la missione può iniziare in qualunque giorno dell'anno
ma deve durare esattamente il numero di giorni indicato e terminare
non oltre la data di scadenza.

Topolino, presa la lista delle missioni ricevuta da Basettoni, ordina
tali missioni in base alla loro data di scadenza.  Quindi, numera i
giorni dell'anno da 1 a 365 (non esistono anni bisestili a Topolinia)
e trasforma le date di scadenza in numeri secondo tale numerazione.
Per esempio, se una missione dura 15 giorni e deve essere svolta entro
il 18 febbraio, Topolino la vede semplicemente come una coppia di
interi 15  49 (in quanto il 18 febbraio è il quarantanovesimo
giorno dell'anno).

Poiché può svolgere una sola missione alla volta,
Topolino sa che potrebbe svolgerne solo alcune pur iniziando una
missione il giorno immediatamente successivo a quello in cui termina
la precedente missione. Vuole perciò sapere il numero massimo
di missioni che è in grado di svolgere rispettando i vincoli
sulla loro durata e scadenza.  Supponendo che Topolino già
fornisca le coppie di interi ordinate per scadenza (il secondo membro
delle coppie), aiutatelo a calcolare il massimo numero di missioni che
può svolgere.

Per esempio, se ci sono quattro missioni, una di tre giorni da
terminare entro il 5 gennaio, una di quattro giorni entro l'8 gennaio,
una di tre giorni entro il 9 gennaio e una di 6 giorni entro il 12
gennaio, Topolino vi fornisce la lista di quattro coppie 3 5, 4 8, 3 9
e 6 12. Il numero massimo di missioni che può svolgere è
pari a tre, ossia le missioni corrispondenti alle coppie 3 5, 3 9 e 6
12: la prima missione inizia il primo di gennaio e termina il 3
gennaio; la seconda inizia il 4 gennaio e termina il 6 gennaio; la
terza inizia il 7 gennaio e termina il 12 gennaio. (Notare che,
scegliendo la missione corrispondente alla coppia 4 8, Topolino
può svolgere al più due missioni.)


\section*{Dati di input}
  Il file \texttt{input.txt} è composto da $N$+1
righe.
La prima riga contiene un intero positivo che rappresenta il numero
$N$ di missioni presentate da Basettoni a Topolino.

Le successive $N$ righe rappresentano durata e scadenza delle
missioni: ciascuna riga è composta da due interi $d$ e
$s$ separati da uno spazio, a rappresentare che la
corrispondente missione dura $d$ giorni e deve essere
completata entro l'$s$-esimo giorno dell'anno.


\section*{Dati di output}
  
Il file \texttt{output.txt} è composto da una sola riga
contenente un intero che rappresenta il massimo numero di missioni che
Topolino può svolgere rispettando i vincoli su durata e scadenza.

  \section*{Assunzioni}
  \begin{itemize}
  
    \item  1 ≤ $N$ ≤ 100.
    \item  1 ≤ $d$, $s$ ≤ 365.
  \end{itemize}

\section*{Esempi di input/output}

  
    \noindent
    \begin{tabular}{p{11cm}|p{5cm}}
    \toprule
    \textbf{File \texttt{input.txt}}
    & \textbf{File \texttt{output.txt}}
    \\
    \midrule
    \scriptsize
    \begin{verbatim}
4
3 5
4 8
3 9
6 12
\end{verbatim}
    &
    \scriptsize
    \begin{verbatim}
3
\end{verbatim}
    \\
    \bottomrule
    \end{tabular}
  
\section*{Nota/e}
\begin{itemize}
  
    \item Un programma che restituisce sempre lo stesso valore,
indipendentemente dai dati in \texttt{input.txt}, non totalizza
alcun punteggio.
\end{itemize}



\end{document}
