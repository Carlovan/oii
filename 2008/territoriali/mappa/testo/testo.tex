
\documentclass[a4paper,11pt]{article}

\usepackage[utf8x]{inputenc}
\SetUnicodeOption{mathletters}
\SetUnicodeOption{autogenerated}

\usepackage[italian]{babel}
\usepackage{booktabs}
\usepackage{mathpazo}
\usepackage{graphicx}
\usepackage[left=2cm, right=2cm, bottom=3cm]{geometry}
\frenchspacing

\begin{document}
\noindent {\Large Selezioni regionali 2008}
\vspace{0.5cm}

\noindent {\Huge Mappa antica (\texttt{mappa})}


\vspace{0.5cm}
\noindent {\Large Difficoltà D = 2.}

\section*{Descrizione del problema}
   
Topolino è in missione per accompagnare una spedizione
archeologica che segue un'antica mappa acquisita di recente dal museo
di Topolinia. Raggiunta la località dove dovrebbe trovarsi un
prezioso e raro reperto archeologico, Topolino si imbatte in un
labirinto che ha la forma di una gigantesca scacchiera quadrata di
$N$x$N$ lastroni di marmo. 

Nella mappa, sia le righe che le colonne del labirinto sono numerate
da 1 a $N$. Il lastrone che si trova nella posizione
corrispondente alla riga $r$ e alla colonna $c$
viene identificato mediante la coppia di interi ($r$,
$c$). I lastroni segnalati da una crocetta '+' sulla mappa
contengono un trabocchetto mortale e sono quindi da evitare, mentre i
rimanenti sono innocui e segnalati da un asterisco '*'.

Topolino deve partire dal lastrone in posizione (1, 1) e
raggiungere il lastrone in posizione ($N$, $N$),
entrambi innocui. Può passare da un lastrone a un
altro soltanto se questi condividono un lato o uno spigolo (quindi
può procedere in direzione orizzontale, verticale o diagonale
ma non saltare) e, ovviamente, questi lastroni devono essere innocui.

Tuttavia, le insidie non sono finite qui: per poter attraversare
incolume il labirinto, Topolino deve calpestare il minor numero
possibile di lastroni innocui (e ovviamente nessun lastrone con
trabocchetto). Aiutate Topolino a calcolare tale numero minimo.


\section*{Dati di input}
  Il file \texttt{input.txt} è composto da $N$+1
righe.
La prima riga contiene un intero positivo che rappresenta la dimensione
$N$ di un lato del labirinto a scacchiera.

Le successive $N$ righe rappresentano il labirinto a
scacchiera: la $r$-esima di tali righe contiene una sequenza
di $N$ caratteri '+' oppure '*', dove '+' indica un lastrone
con trabocchetto mentre '*' indica un lastrone sicuro. Tale riga
rappresenta quindi i lastroni che si trovano sulla $r$-esima
riga della schacchiera: di conseguenza, il $c$-esimo
carattere corrisponde al lastrone in posizione ($r$, $c$).


\section*{Dati di output}
  
Il file \texttt{output.txt} è composto da una sola riga
contenente un intero che rappresenta il minimo numero di lastroni
innocui (ossia indicati con '*') che Topolino deve attraversare a
partire dal lastrone in posizione (1, 1) per arrivare incolume al
lastrone in posizione ($N$, $N$). Notare che i
lastroni (1, 1) e ($N$, $N$) vanno inclusi nel
conteggio dei lastroni attraversati.

  \section*{Assunzioni}
  \begin{itemize}
  
    \item  1 ≤ $N$ ≤ 100.
    \item  1 ≤ $r$, $c$ ≤ $N$.
    \item  E' sempre possibile attraversare il labirinto dal lastrone in
posizione (1, 1) al lastrone in posizione ($N$, $N$);
inoltre tali due lastroni sono innocui.
  \end{itemize}

\section*{Esempi di input/output}

  
    \noindent
    \begin{tabular}{p{11cm}|p{5cm}}
    \toprule
    \textbf{File \texttt{input.txt}}
    & \textbf{File \texttt{output.txt}}
    \\
    \midrule
    \scriptsize
    \begin{verbatim}
4
*+++
+**+
+*+*
+***
\end{verbatim}
    &
    \scriptsize
    \begin{verbatim}
5
\end{verbatim}
    \\
    \bottomrule
    \end{tabular}
  
\section*{Nota/e}
\begin{itemize}
  
    \item Un programma che restituisce sempre lo stesso valore,
indipendentemente dai dati in \texttt{input.txt}, non totalizza
alcun punteggio.
\end{itemize}



\end{document}
