
\documentclass[a4paper,11pt]{article}

\usepackage[utf8x]{inputenc}
\SetUnicodeOption{mathletters}
\SetUnicodeOption{autogenerated}

\usepackage[italian]{babel}
\usepackage{booktabs}
\usepackage{mathpazo}
\usepackage{graphicx}
\usepackage[left=2cm, right=2cm, bottom=3cm]{geometry}
\frenchspacing

\begin{document}
\noindent {\Large Selezioni regionali 2008}
\vspace{0.5cm}

\noindent {\Huge Codici e pizzini (\texttt{pizzini})}


\vspace{0.5cm}
\noindent {\Large Difficoltà D = 1.}

\section*{Descrizione del problema}
   
Il Commissario Basettoni è riuscito a localizzare il
nascondiglio del pericoloso Gambadilegno. Facendo irruzione nel covo,
Basettoni trova una serie di foglietti (detti "pizzini") che
riportano, cifrati, i codici di accesso ai conti correnti del
gruppo di malavitosi capeggiato da Gambadilegno.

Il Commissario Basettoni chiede aiuto a Topolino per interpretare
questi pizzini. Dopo approfondite analisi, Topolino scopre le seguenti
cose:

\begin{itemize}
  
    \item ogni pizzino contiene $N$ righe e ciascuna riga è
una sequenza di cifre decimali ('0', '1', ..., '9') concatenate senza
spazi intermedi (quindi la sequenza 0991, come tale, non va
interpretata come il numero 991);

    \item ogni pizzino riporta, cifrato, un codice di accesso a $N$
cifre;

    \item tale codice si ottiene concatenando una dopo l'altra, senza spazi
intermedi, le cifre estratte dalle $N$ sequenze scritte nel
pizzino, più precisamente, una cifra per ogni sequenza;

    \item la cifra da estrarre per ciascuna sequenza è quella in
posizione $p$, dove $p$ è il numero di
anagrammi che, per tale sequenza, appaiono nel pizzino.

\end{itemize}

Un anagramma di una sequenza $S$ è ottenuto permutando
le sue cifre (per esempio, 1949 e 9419 sono anagrammi); inoltre,
$S$ è anagramma di se stessa.  Quindi Topolino deduce
che, per calcolare il numero $p$ di anagrammi di
$S$, deve includere $S$ tra i suoi anagrammi
contenuti nel pizzino. In questo modo, $p$ = 1 indica che una
sequenza non ha altri anagrammi, a parte se stessa, per cui va
estratta la sua prima cifra.

Per illustrare quanto descritto sopra a Basettoni, Topolino prende un
pizzino che contiene i tre anagrammi 1949, 9419 e 9149 (e non ce ne
sono altri) e ne estrae la loro terza cifra, ossia 4, 1 e 4,
poiché $p$ = 3; poi, prende un altro pizzino con due
soli anagrammi 1949 e 9419, estraendone la seconda cifra, ossia 9 e 4,
poiché $p$ = 2. Utilizzando questo meccanismo di
estrazione delle cifre, aiutate Topolino a decifrare i pizzini di
Gambadilegno trovati da Basettoni.


\section*{Dati di input}
  Il file \texttt{input.txt} è composto da $N$+1
righe.
La prima riga contiene un intero positivo che rappresenta il numero
$N$ di sequenze contenute nel pizzino.

Ciascuna delle successive $N$ righe contiene una sequenza di
cifre decimali ('0', '1', ..., '9') senza spazi intermedi. 


\section*{Dati di output}
  
Il file \texttt{output.txt} è composto da una sola riga
contenente una sequenza di $N$ cifre decimali, senza spazi
intermedi, ossia il codice di accesso cifrato nel pizzino.

  \section*{Assunzioni}
  \begin{itemize}
  
    \item  1 ≤ $N$ ≤ 100.
    \item  Ogni sequenza contiene al massimo 80 cifre decimali.
    \item  Le sequenze contenute in uno stesso pizzino sono tutte diverse
tra di loro.
    \item  Una sequenza di $K$ cifre decimali presenta al massimo
$K$ anagrammi in uno stesso pizzino. Inoltre, tali anagrammi
non necessariamente appaiono in righe consecutive del pizzino.
  \end{itemize}

\section*{Esempi di input/output}

  
    \noindent
    \begin{tabular}{p{11cm}|p{5cm}}
    \toprule
    \textbf{File \texttt{input.txt}}
    & \textbf{File \texttt{output.txt}}
    \\
    \midrule
    \scriptsize
    \begin{verbatim}
6
1949
21
9419
12
4356373
9149
\end{verbatim}
    &
    \scriptsize
    \begin{verbatim}
411244
\end{verbatim}
    \\
    \bottomrule
    \end{tabular}
  
    \noindent
    \begin{tabular}{p{11cm}|p{5cm}}
    \toprule
    \textbf{File \texttt{input.txt}}
    & \textbf{File \texttt{output.txt}}
    \\
    \midrule
    \scriptsize
    \begin{verbatim}
4
022
524
322
742
\end{verbatim}
    &
    \scriptsize
    \begin{verbatim}
0537
\end{verbatim}
    \\
    \bottomrule
    \end{tabular}
  
\section*{Nota/e}
\begin{itemize}
  
    \item Un programma che restituisce sempre lo stesso valore,
indipendentemente dai dati in \texttt{input.txt}, non totalizza
alcun punteggio.
\end{itemize}



\end{document}
