
\documentclass[a4paper,11pt]{article}

\usepackage[utf8x]{inputenc}
\SetUnicodeOption{mathletters}
\SetUnicodeOption{autogenerated}

\usepackage[italian]{babel}
\usepackage{booktabs}
\usepackage{mathpazo}
\usepackage{graphicx}
\usepackage[left=2cm, right=2cm, bottom=3cm]{geometry}
\frenchspacing

\begin{document}
\noindent {\Large Selezioni nazionali 2010}
\vspace{0.5cm}

\noindent {\Huge Esercizio 2: Numeri antipatici (\texttt{numeri})}


\vspace{0.5cm}
\noindent {\Large Difficoltà D = 2 (tempo limite 1 sec).}

\section*{Descrizione del problema}
  
E' ben nota l'antipatia umorale della Regina di Cuori per certi
numeri: per esempio, odia il numero 13; non solo, anche il 17 non
è molto amato da Sua Maestà. Ahimè, questi non
sono gli unici casi poiché ci sono diversi numeri $M$
che non sono tollerati dalla Regina e a farne le spese sono i poveri
giardinieri. Il problema è che, ogni mattina, la Regina si alza
e indica ai giardinieri qual è il numero $M$ che le
è antipatico quel giorno.

Lungo il dritto viale che porta alla regale dimora, c'è un 
filare di $N$ piante di rose. Purtroppo, la Regina conta le rose mentre
passeggia nel viale e non sopporta che una sequenza di una o
più piante consecutive contenga un totale di $M$ rose:
ha fatto tagliare diverse teste per questioni meno gravi.

I giardinieri sono terrorizzati dal fatto che lo Stregatto ci abbia
messo lo zampino, alterando il numero di rose in modo da far
apparire $M$ rose. Aiutali a
individuare il numero $S$ di sequenze le cui piante
totalizzano $M$ rose nel filare. Notare che alcune piante
possono contenere zero rose.

Per esempio, con un filare di $N$=20 piante, contenenti
rispettivamente 2, 3, 0, 4, 0, 3, 1, 0, 1, 0, 1, 0, 0, 0, 5, 0, 4, 0,
0, 2 rose, il numero
$M$=9 appare in $S$=18 sequenze:\\
\fbox{2, 3, 0, 4}, 0, 3, 1, 0, 1, 0, 1, 0, 0, 0, 5, 0, 4, 0, 0, 2\\ 
\fbox{2, 3, 0, 4, 0}, 3, 1, 0, 1, 0, 1, 0, 0, 0, 5, 0, 4, 0, 0, 2\\ 
2, 3, \fbox{0, 4, 0, 3, 1, 0, 1}, 0, 1, 0, 0, 0, 5, 0, 4, 0, 0, 2\\ 
2, 3, \fbox{0, 4, 0, 3, 1, 0, 1, 0}, 1, 0, 0, 0, 5, 0, 4, 0, 0, 2\\ 
2, 3, 0, \fbox{4, 0, 3, 1, 0, 1}, 0, 1, 0, 0, 0, 5, 0, 4, 0, 0, 2\\ 
2, 3, 0, \fbox{4, 0, 3, 1, 0, 1, 0}, 1, 0, 0, 0, 5, 0, 4, 0, 0, 2\\ 
2, 3, 0, 4, 0, 3, 1, 0, 1, 0, 1, \fbox{0, 0, 0, 5, 0, 4}, 0, 0, 2\\ 
2, 3, 0, 4, 0, 3, 1, 0, 1, 0, 1, 0, \fbox{0, 0, 5, 0, 4}, 0, 0, 2\\ 
2, 3, 0, 4, 0, 3, 1, 0, 1, 0, 1, 0, 0, \fbox{0, 5, 0, 4}, 0, 0, 2\\ 
2, 3, 0, 4, 0, 3, 1, 0, 1, 0, 1, 0, 0, 0, \fbox{5, 0, 4}, 0, 0, 2\\ 
2, 3, 0, 4, 0, 3, 1, 0, 1, 0, 1, \fbox{0, 0, 0, 5, 0, 4, 0}, 0, 2\\ 
2, 3, 0, 4, 0, 3, 1, 0, 1, 0, 1, 0, \fbox{0, 0, 5, 0, 4, 0}, 0, 2\\ 
2, 3, 0, 4, 0, 3, 1, 0, 1, 0, 1, 0, 0, \fbox{0, 5, 0, 4, 0}, 0, 2\\ 
2, 3, 0, 4, 0, 3, 1, 0, 1, 0, 1, 0, 0, 0, \fbox{5, 0, 4, 0}, 0, 2\\ 
2, 3, 0, 4, 0, 3, 1, 0, 1, 0, 1, \fbox{0, 0, 0, 5, 0, 4, 0, 0}, 2\\ 
2, 3, 0, 4, 0, 3, 1, 0, 1, 0, 1, 0, \fbox{0, 0, 5, 0, 4, 0, 0}, 2\\ 
2, 3, 0, 4, 0, 3, 1, 0, 1, 0, 1, 0, 0, \fbox{0, 5, 0, 4, 0, 0}, 2\\ 
2, 3, 0, 4, 0, 3, 1, 0, 1, 0, 1, 0, 0, 0, \fbox{5, 0, 4, 0, 0}, 2\\ 


\section*{Dati di input}
  Il file \texttt{input.txt} è composto da due righe.

La prima riga contiene due interi $M$ e $N$ separati
da uno spazio: $M$ è il numero antipatico alla
Regina di Cuori quel giorno, e $N$ è il numero di piante lungo il
filare che costeggia il viale.

La seconda riga contiene $N$ interi positivi separati da uno
spazio: l'$I$-esimo intero indica il numero di rose
nella  $I$-esima pianta nel filare.


\section*{Dati di output}
  
Il file \texttt{output.txt} è composto da una sola riga che
contiene l'intero positivo $S$, a indicare il numero di
sequenze di piante nel filare che totalizzano $M$ rose.

  \section*{Assunzioni}
  \begin{itemize}
  
    \item  0 ≤ $M < 2^{31}$
    \item  1 ≤ $N$ ≤ 1 000 000
    \item  1 ≤ $S < 2^{31}$ (maledetto Stregatto!).
    \item  0 ≤ $I < 2^{31}$,
  dove $I$ è il numero di rose in una pianta.
  \end{itemize}

\section*{Esempi di input/output}

  
    \noindent
    \begin{tabular}{p{11cm}|p{5cm}}
    \toprule
    \textbf{File \texttt{input.txt}}
    & \textbf{File \texttt{output.txt}}
    \\
    \midrule
    \scriptsize
    \begin{verbatim}
9 20
2 3 0 4 0 3 1 0 1 0 1 0 0 0 5 0 4 0 0 2 
\end{verbatim}
    &
    \scriptsize
    \begin{verbatim}
18
\end{verbatim}
    \\
    \bottomrule
    \end{tabular}
  
    \noindent
    \begin{tabular}{p{11cm}|p{5cm}}
    \toprule
    \textbf{File \texttt{input.txt}}
    & \textbf{File \texttt{output.txt}}
    \\
    \midrule
    \scriptsize
    \begin{verbatim}
0 5
0 0 0 0 0
\end{verbatim}
    &
    \scriptsize
    \begin{verbatim}
15
\end{verbatim}
    \\
    \bottomrule
    \end{tabular}
  
\section*{Nota/e}
\begin{itemize}
  
    \item  
Una sequenza può essere composta anche da una sola pianta, se
quest'ultima contiene esattamente $M$ rose.

\end{itemize}



\end{document}
