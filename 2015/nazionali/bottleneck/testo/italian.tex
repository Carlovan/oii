\usepackage{xcolor}
\usepackage{afterpage}
\usepackage{pifont,mdframed}
\usepackage[bottom]{footmisc}
\usepackage{subcaption}

\createsection{\Grader}{Grader di prova}

\makeatletter
\gdef\this@inputfilename{input.txt}
\gdef\this@outputfilename{output.txt}
\makeatother

\newenvironment{warning}
  {\par\begin{mdframed}[linewidth=2pt,linecolor=gray]%
    \begin{list}{}{\leftmargin=1cm
                   \labelwidth=\leftmargin}\item[\Large\ding{43}]}
  {\end{list}\end{mdframed}\par}

\newcommand{\inputfile}{\texttt{input.txt}}
\newcommand{\outputfile}{\texttt{output.txt}}

% Collo di bottiglia (bottleneck)

Luca e William stanno valutando una nuova e apparentemente vantaggiosa offerta della Fraudolent, la più diffusa compagnia telefonica dell'arcipelago delle isole NoWhere. Il contratto garantisce ai suoi utilizzatori che ogni comunicazione online tra due computer seguirà sempre il percorso più breve (in termini di nodi intermedi) all'interno della rete internet dell'arcipelago.

Dati i trascorsi della società con la giustizia, tuttavia, i due ragazzi sono sospettosi, e temono che quello della Fraudolent sia solo un torbido tentativo di ingannare gli utenti più incauti: non a caso, infatti, il contratto parla di connessione \emph{più breve}, e mai di più veloce. Per fugare ogni dubbio, Luca e William decidono di valutare, nel caso in cui dovessero sottoscrivere la promozione, la velocità di trasmissione dei dati tra i propri computer.

Per misurare la minima velocità di trasmissione garantita dal contratto, i ragazzi hanno mappato l'intera rete internet dell'arcipelago. Questa è rappresentata da un grafo in cui i vertici costituiscono i vari nodi della rete (tra cui i computer di William e di Luca) e gli archi identificano i collegamenti tra questi. Gli archi riportano anche la propria \emph{capacità}, cioè il massimo numero di megabit che ogni secondo possono fluire attraverso di essi. La \emph{velocità di trasmissione} in un percorso è pari alla minima capacità degli archi che lo compongono.

William e Luca sanno che la Fraudolent, pur essendo vincolata da contratto a garantire una trasmissione che passi per il minimo numero possibile di nodi intermedi, sceglierà sempre, tra tutti, il percorso più lento per collegare due computer, per risparmiare sull'utilizzo della rete. 

Aiuta Luca e William a determinare, data la mappa della rete dell'arcipelago, quale sarebbe la velocità di trasmissione dati tra i propri computer. Per esempio, consideriamo la rete in figura~\ref{fig:esempio}: il computer di William corrisponde al nodo 2, colorato di blu, mentre quello di Luca al nodo 8, colorato di rosso; i numeri sui collegamenti rappresentano la capacità degli archi, in megabit al secondo.

\begin{figure}[h!]
	\centering\hspace*{2.5cm}\includegraphics[scale = .75]{asy_bottleneck/esempio.pdf}
	\caption{\label{fig:esempio}La rete dell'arcipelago.}
\end{figure}

I tre percorsi di lunghezza minima tra i due computer sono mostrati qui sotto. La velocità della connessione in questo caso sarebbe pari a 2 megabit al secondo, corrispondente alla velocità di trasmissione del secondo percorso.

\begin{figure}[H]
  \centering
  \begin{subfigure}[c]{0.30\textwidth}
    \centering\includegraphics[width=.95\textwidth]{asy_bottleneck/esempio_2.pdf}
    \label{fig:percorso-1}
  \end{subfigure}
  \hfill
  \begin{subfigure}[c]{0.30\textwidth}
    \centering\includegraphics[width=.95\textwidth]{asy_bottleneck/esempio_3.pdf}
    \label{fig:percorso-2}
  \end{subfigure}
  \hfill
  \begin{subfigure}[c]{0.30\textwidth}
    \centering\includegraphics[width=.95\textwidth]{asy_bottleneck/esempio_4.pdf}
    \label{fig:percorso-3}
  \end{subfigure}
\end{figure}

\Scoring
Il tuo programma verrà testato su diversi test case raggruppati in subtask.
Per ottenere il punteggio relativo ad un subtask, è necessario risolvere
correttamente tutti i test relativi ad esso.

\begin{itemize}[nolistsep, itemsep=2mm]
    \item \textbf{\makebox[2cm][l]{Subtask 1} [5 punti]:} Caso d'esempio
    \item \textbf{\makebox[2cm][l]{Subtask 2} [7 punti]:} I nodi di rete sono collegati in sequenza, come in figura
	    \begin{figure}[H]
		    \centering\includegraphics[scale = .77]{asy_bottleneck/linea.pdf}
	    \end{figure}
	    È sempre garantito che il primo nodo della sequenza è il nodo 1, e che il nodo $k$ segue sempre il nodo $k-1$, per ogni $k \ge 2$.
    \item \textbf{\makebox[2cm][l]{Subtask 3} [10 punti]:} I nodi di rete sono collegati ad anello, come in figura
	    \begin{figure}[H]
		    \centering\includegraphics[scale = .77]{asy_bottleneck/anello.pdf}
	    \end{figure}
	    È sempre garantito che il nodo $N$ è collegato al nodo 1, e che il nodo $k$ segue il nodo $k-1$, per ogni $2 \le k \le N$.
    \item \textbf{\makebox[2cm][l]{Subtask 4} [11 punti]:} I nodi di rete sono collegati a griglia, come in figura
	    \begin{figure}[H]
		    \centering\includegraphics[scale = .77]{asy_bottleneck/griglia.pdf}
	    \end{figure}
	    È sempre garantito che i nodi seguiranno una numerazione per righe.
    \item \textbf{\makebox[2cm][l]{Subtask 5} [17 punti]:} Tra ogni coppia di nodi di rete esiste un unico percorso che li collega, come in figura
	    \begin{figure}[H]
		    \centering\includegraphics[scale = .77]{asy_bottleneck/albero.pdf}
	    \end{figure}
    \item \textbf{\makebox[2cm][l]{Subtask 6} [22 punti]:} $N, M\leq 1000$.
    \item \textbf{\makebox[2cm][l]{Subtask 7} [28 punti]:} Nessuna limitazione specifica (vedi la sezione \textbf{Assunzioni}).
\end{itemize}

\newpage
\Implementation
Dovrai sottoporre esattamente un file con estensione \texttt{.c}, \texttt{.cpp} o \texttt{.pas}.

\begin{warning}
Tra gli allegati a questo task troverai un template (\texttt{bottleneck.c}, \texttt{bottleneck.cpp}, \texttt{bottleneck.pas}) con un esempio di implementazione.
\end{warning}

Dovrai implementare la seguente funzione:

\begin{center}\begin{tabular}{|c|l|}
\hline
C/C++    & \begin{minipage}{.87\textwidth}
\begin{verbatim}
int Analizza(int N, int M, int W, int L,
             int arco_da[], int arco_a[], int capacita[], int R, int C);
\end{verbatim}
\end{minipage}\\
\hline
Pascal & \begin{minipage}{.87\textwidth}
\begin{verbatim}
function Analizza(N, M, W, L: longint;
                  var arco_da, arco_a, capacita: array of longint;
                  R, C: longint): longint;
\end{verbatim}
\end{minipage}\\
\hline
\end{tabular}\end{center}
dove:
\begin{itemize}[nolistsep, itemsep=2mm]
	\item $N$ rappresenta il numero di nodi di rete.
	\item $M$ rappresenta il numero di collegamenti.
	\item $W$ e $L$ sono rispettivamente il computer di William e quello di Luca.
	\item \texttt{arco\_da} e \texttt{arco\_a} sono due array di dimensione $M$ che rappresentano i collegamenti. L'$i$-esimo collegamento di rete connette (in modo bidirezionale) i nodi \texttt{arco\_da}$[i]$ e \texttt{arco\_a}$[i]$. È garantito che lo stesso collegamento non venga mai ripetuto.
	\item \texttt{capacita} è un array di dimensione $M$. L'intero \texttt{capacita}$[i]$ rappresenta la capacità dell'$i$-esimo collegamento, in megabit al secondo.
	\item $R$ e $C$ sono parametri speciali che di norma valgono $-1$. L'unica eccezione è il caso della topologia a griglia (vedi \textbf{Subtask 4}), in cui $R$ e $C$ rappresentano rispettivamente il numero di righe e di colonne della griglia.
\end{itemize}

\Grader
Nella directory relativa a questo problema è presente una versione 
semplificata del grader usato durante la correzione, che potete usare
per testare le vostre soluzioni in locale. Il grader di esempio legge
i dati di input dal file \texttt{input.txt}, a quel punto chiama la
funzione \texttt{Analizza} che dovete implementare. Il grader scrive sul file \outputfile{} la risposta fornita dalla funzione \texttt{Analizza}.

Nel caso vogliate generare un input per un test di valutazione, il file \inputfile{} deve avere questo formato:

\begin{itemize}[nolistsep,itemsep=2mm]
\item Riga $1$: contiene l'intero \texttt{N}, che rappresenta il numero di nodi di rete, l'intero \texttt{M}, che rappresenta il numero di collegamenti, gli interi $W$ e $L$, che sono i computer di William e di Luca rispettivamente, gli interi $R$ e $C$, che di norma valgono -1 e nel caso della topologia a griglia rappresentano rispettivamente il numero di righe e di colonne della griglia.
\item Righe $2, \dots, M+1$: l'$i$-esima riga contiene tre interi $x_i$, $y_i$, $z_i$ con $1\le x_i\le N$ e $1\le y_i\le N$, che rappresentano un collegamento tra i nodi $x_i$ e $y_i$ con capacità $z_i$.
\end{itemize}

Il file \outputfile{} invece ha questo formato:
\begin{itemize}[nolistsep,itemsep=2mm]
\item Riga $1$: contiene il valore restituito dalla funzione \texttt{Analizza}.
\end{itemize}

\newpage
\Constraints 
\begin{itemize}[nolistsep, itemsep=2mm]
	\item $2 \le N \le 100\,000$.
	\item $1 \le M \le 1\,000\,000$.
	\item Le capacità dei collegamenti sono interi compresi tra $0$ e $1\,000\,000\,000$.
	\item Quando $R$ e $C$ non valgono entrambi $-1$, vale che $1 \le R,C \le 300$.
	\item Il grafo della rete è connesso, cioè ogni nodo è raggiungibile da tutti gli altri.
	\item Nessun collegamento connette un nodo con se stesso.
	\item Per ogni coppia di nodi c'è al più un collegamento che li connette.
\end{itemize}

\Examples
\begin{example}
\exmp{
12 17 2 8 3 4
1 2 1
2 3 8
3 4 3
1 5 3
2 6 7
3 7 2
4 8 9
5 6 4
5 9 6
6 7 5
7 8 6
9 10 1
10 11 2
11 12 5
10 6 8
11 7 5
8 12 2
}{%
2
}%
\end{example}

Questo caso di input corrisponde all'esempio spiegato nel testo.

\newpage
\setcounter{figure}{0}
\definecolor{backcolor}{gray}{0.95}
\pagecolor{backcolor}
\createsection{\Solution}{Soluzione}
\createsection{\SolDueBfs}{{\small{$\blacksquare$}} \normalsize Secondo approccio}
\createsection{\SolUnaBfs}{{\small{$\blacksquare$}} \normalsize Primo approccio}

\Solution
Questo problema ammette diverse soluzioni ``specializzate'', in grado di risolvere solo alcuni dei subtask proposti. Non ci occuperemo di tali soluzioni parziali in questa sede; mostreremo invece due approcci, diversi ma equivalenti dal punto di vista della correttezza, per risolvere completamente il problema. 

\SolUnaBfs
\begin{wrapfigure}{r}{.35\textwidth}
  \vspace*{-.8cm}
  \begin{flushright}
		\includegraphics[width=.34\textwidth]{asy_bottleneck/sol1.pdf}
		\caption{Mappa d'esempio.}
		\label{fig:fig1}
	\end{flushright}
	\vspace*{-1cm}
\end{wrapfigure}
La prima delle due soluzioni che proponiamo consiste in una versione modificata dell'algoritmo di \emph{visita in ampiezza} (BFS) di un grafo. Intuitivamente, partiamo dal PC di William, e visitiamo tutti i nodi a distanza 1, 2, $\dots$ fino a raggiungere il computer di Luca.
Mano a mano che ci espandiamo nel grafo, manteniamo per ogni nodo $v$ la minima capacità del percorso tra il computer di William e $v$ stesso.

Illustriamo i dettagli dell'algoritmo attraverso un esempio: supponiamo che la mappa della rete coincida con quella di Figura~\ref{fig:fig1}, dove il nodo blu ancora una volta corrisponde al PC di William (d'ora in poi indicato con $W$) e il nodo rosso a quello di Luca (d'ora in poi indicato con $L$).

\begin{enumerate}[nolistsep, itemsep=2mm, leftmargin=.8cm]
	\item Come prima cosa, per ogni nodo $v$ memorizziamo due valori: $\texttt{distanza}[v]$ indica la distanza tra il nodo $v$ e il nodo $W$, ovvero il minimo numero di archi necessari per raggiungere $v$ a partire da $W$, mentre $\texttt{velocità}[v]$ indica la minima velocità di trasmissione tra tutti i percorsi di lunghezza minima da $W$ a $v$. Inizialmente segniamo che $\texttt{distanza}[v] = \texttt{velocità}[v] = \infty$ per ogni nodo $v \neq W$. Per quanto riguarda il nodo $W$, invece, segniamo $\texttt{distanza}[W] = 0$ e 
	$\texttt{velocità}[W] = \infty$. Al termine dell'esecuzione dell'algoritmo tutti i nodi avranno i valori \texttt{distanza} e \texttt{velocità} corretti. La risposta al problema sarà quindi $\texttt{velocità}[L]$.
	
	\item Consideriamo i nodi \emph{in ordine di distanza da $W$}, mano a mano che ci espandiamo. Supponiamo di star considerando il nodo $v$. Per ogni vicino $w$ di $v$ sia $c$ la capacità dell'arco $v - w$; se risulta che $\texttt{distanza}[w] > \texttt{distanza}[v]$ eseguiamo il doppio assegnamento $\texttt{distanza}[w] = \texttt{distanza}[v] + 1$, $\texttt{velocità}[w] = \min\{\texttt{velocità}[w], \texttt{velocità}[v], c\}$.
\end{enumerate}
\vfill
\captionsetup[subfigure]{labelformat=empty}
\begin{figure}[H]
  \begin{subfigure}[c]{0.20\textwidth}
  	\includegraphics[width=.99\textwidth]{asy_bottleneck/sol2.pdf}
  	\caption{(2a)}
  \end{subfigure}\hfill
  \begin{subfigure}[c]{0.20\textwidth}
\includegraphics[width=.99\textwidth]{asy_bottleneck/sol3a.pdf}
	\caption{(2b)}
  \end{subfigure}\hfill
  \begin{subfigure}[c]{0.20\textwidth}
\includegraphics[width=.99\textwidth]{asy_bottleneck/sol3b.pdf}
	\caption{(2c)}
  \end{subfigure}\hfill
  \begin{subfigure}[c]{0.20\textwidth}
\includegraphics[width=.99\textwidth]{asy_bottleneck/sol4a.pdf}
	\caption{(2d)}
  \end{subfigure}\\[3mm]
  \begin{subfigure}[c]{0.20\textwidth}
\includegraphics[width=.99\textwidth]{asy_bottleneck/sol4b.pdf}
	\caption{(2e)}
  \end{subfigure}\hfill
  \begin{subfigure}[c]{0.20\textwidth}
\includegraphics[width=.99\textwidth]{asy_bottleneck/sol4c.pdf}
	\caption{(2f)}
  \end{subfigure}\hfill
  \begin{subfigure}[c]{0.20\textwidth}
\includegraphics[width=.99\textwidth]{asy_bottleneck/sol5a.pdf}
	\caption{(2g)}
  \end{subfigure}\hfill
  \begin{subfigure}[c]{0.20\textwidth}
\includegraphics[width=.99\textwidth]{asy_bottleneck/sol6.pdf}
	\caption{(2h)}
  \end{subfigure}
\end{figure}
Le figure (2a) -- (2h) mostrano alcuni passi dell'algoritmo (sono state omesse alcune coppie di nodi per cui non valeva la disuguaglianza $\texttt{distanza}[w] > \texttt{distanza}[v]$). Per dimostrare che l'algoritmo è a tutti gli effetti corretto, cominciamo col notare che non appena il valore $\texttt{distanza}$ di un nodo risulta diverso da $\infty$, esso rappresenta effettivamente la distanza, in termini di archi, tra il nodo $W$ e il nodo $v$. Infatti, se immaginiamo di omettere completamente l'informazione $\texttt{velocità}$, l'algoritmo coincide con quello della \emph{visita in ampiezza} di un grafo, la cui correttezza viene qui data per assodata. Notiamo inoltre che dal momento che il grafo è connesso, al termine dell'esecuzione dell'algoritmo ogni nodo è stato processato.

Rimane da dimostrare che al termine dell'algoritmo le informazioni \texttt{velocità} risultano corrette per ogni nodo\footnote{Si assume corretto per $W$ il valore fittizio $\texttt{velocità}[W] = \infty$.}. A tal fine, possiamo procedere per induzione sulla frontiera dei nodi: supponiamo di aver già processato tutti i nodi a distanza $< k$ da $W$, e che l'informazione $\texttt{velocità}$ trovate per tali nodi sia corretta: la tesi vale per la base di induzione $k = 1$, ovvero per il singolo nodo $W$. Vogliamo ora processare tutti i nodi a distanza esattamente $k$, e dimostrare che l'algoritmo attribuisce a tutti questi un valore \texttt{velocità} effettivamente corretto.

Sia $v$ uno qualunque dei nodi a distanza $k$ da $W$: dimostriamo che al termine dell'esecuzione dell'algoritmo $\texttt{velocità}[v]$ contiene effettivamente la minima velocità di trasmissione tra tutti i percorsi composti da $k$ archi tra il nodo $W$ e il nodo $v$. Notiamo che ogni percorso composto da $k$ archi tra i nodi $W$ e $v$ necessariamente prevede come penultimo nodo un nodo $u$ a distanza $k-1$ da $W$. Possiamo allora immaginare di scegliere in tutti i modi possibili l'ultimo arco $u - v$ del percorso, con $u$ nodo a distanza $k-1$ da $W$, e cercare il percorso composto da $k-1$ archi con la velocità di trasmissione minima da $W$ a $u$. Poiché per ipotesi induttiva i valori \texttt{velocità} di tutti i nodi a distanza $k-1$ da $W$ contengono informazioni corrette, possiamo concludere che $$\texttt{velocità}[v] = \min_{u} \{\texttt{velocità}[u], \text{capacità dell'arco } u-v\},$$ dove il minimo va inteso tra tutti i nodi $u$ adiacenti a $v$ e posti a distanza $k-1$ da $W$. In effetti è facile convincersi che questo è esattamente il calcolo eseguito dall'algoritmo.

\SolDueBfs
La seconda soluzione nasce dall'osservazione che è possibile determinare in tempo costante se un particolare arco appartiene ad un percorso di lunghezza minima (in termini di archi) tra il PC di William e quello di Luca. Per ogni nodo $v$ del grafo memorizziamo due valori
\begin{wrapfigure}{r}{.26\textwidth}
  \vspace*{-.5cm}
  \begin{flushright}
		\includegraphics[width=.25\textwidth]{asy_bottleneck/sol_2bfs_1.pdf}\\
		\includegraphics[width=.25\textwidth]{asy_bottleneck/sol_2bfs_2.pdf}\\
	\end{flushright}
	\vspace*{-1.2cm}
\end{wrapfigure}
\texttt{distanza\_da\_W} e \texttt{distanza\_da\_L}: essi rappresentano rispettivamente la distanza tra il computer $W$ di William e il computer $L$ di Luca, ovvero il minimo numero di archi che separano $v$ dal computer indicato. Possiamo precalcolare questi valori tramite due \emph{visite in ampiezza del grafo} (BFS), una a partire da $W$ e una a partire da $L$.

Indichiamo con $d$ la distanza tra i nodi $W$ e $L$: $d$ coincide col valore $\texttt{distanza\_da\_W}[L]$, ovvero col valore $\texttt{distanza\_da\_L}[W]$. È facile convincersi del fatto che l'arco $u - v$ appartiene ad un percorso di lunghezza minima tra $L$ e $W$ se e solo se vale almeno una delle due condizioni:
\begin{itemize}[nolistsep, itemsep=2mm]
	\item $\texttt{distanza\_da\_L}[u] + \texttt{distanza\_da\_W}[v] + 1 = d$, oppure
	\item $\texttt{distanza\_da\_L}[v] + \texttt{distanza\_da\_W}[u] + 1 = d$.
\end{itemize}
Possiamo dunque iterare su tutti gli archi: la risposta al problema è pari alla capacità minima tra le capacità degli archi che appartengono ad almeno un percorso di lunghezza minima tra i nodi $W$ e $L$.

\newpage
\createsection{\Cppsol}{Esempio di codice \texttt{C++11}}
\Cppsol
Proponiamo qui un'implementazione per entrambe le soluzioni.\\[.2cm]
\SolUnaBfs
\colorbox{white}{\makebox[.99\textwidth][l]{\includegraphics[scale=1.2]{code_bottleneck_1.pdf}}}
\newpage
\SolDueBfs
\colorbox{white}{\makebox[.99\textwidth][l]{\includegraphics[trim=3px 0 0 0, clip, scale=1.2]{code_bottleneck_2.pdf}}}

\afterpage{\nopagecolor}
