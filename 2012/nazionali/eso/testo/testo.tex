\documentclass[a4paper,11pt]{article}

\usepackage[utf8x]{inputenc}
\SetUnicodeOption{mathletters}
\SetUnicodeOption{autogenerated}

\usepackage[italian]{babel}
\usepackage{booktabs}
\usepackage{mathpazo}
\usepackage{graphicx}
\usepackage[left=2cm, right=2cm, bottom=3cm]{geometry}
\frenchspacing

\begin{document}
\noindent {\Large Olimpiadi di Informatica: selezioni nazionali 2012}
\vspace{0.5cm}

\noindent {\Huge Espressioni senza operatori (\texttt{eso})}


\section*{Descrizione del problema}

Un'espressione senza operatori (ESO) contiene coppie annidate di
parentesi tonde \texttt{()} oppure quadre \texttt{[]} e sequenze di numeri; inoltre, una
ESO inizia sempre con la tonda aperta e termina sempre con la tonda
chiusa: tale coppia di parentesi racchiude l'intera espressione. Per esempio, la seguente è una ESO: 
\texttt{( 58 [ 65 84 ( 47 ) 29 81 [ ( 18 21 ) 91 72 ] 45 ] 19 12 34 15 )}.

L'ordine degli elementi immediamente contenuti in una coppia di
parentesi (cioè interi o altre coppie di parentesi annidate con tutto
ciò contenuto in esse) può essere cambiato con le seguenti regole, dove
\texttt{a}, \texttt{b}, \texttt{c}, \texttt{d}, \dots, sono numeri oppure
coppie di parentesi che contengono sottoespressioni:

\begin{itemize}
  \item L'espressione \texttt{( a b c d \dots )} può essere permutata in un qualunque ordine, per esempio \texttt{( c d b a \dots )};
  \item L'espressione \texttt{[ a b c d \dots ]} può rimanere con l'ordine corrente \texttt{[ a b c d \dots ]} oppure può essere ribaltata in \texttt{[ \dots\ d c b a ]}.
\end{itemize}


Due ESO sono \emph{equivalenti} se contengono gli stessi elementi e
possono essere trasformate l'una nell'altra utilizzando le due regole
suddette.  Per esempio, la ESO 
\texttt{( 15 19 34 [ 45 [ ( 21 18 ) 91 72 ] 81 29 ( 47 ) 84 65 ] 12 58 )}
è equivalente a quella elencata sopra perché possiamo ottenerla con
la seguente trasformazione:

\begin{itemize}
  \item Il primo passo trasforma \underline{\texttt{( 15 19 34 [ 45 [ ( 21 18 ) 91 72 ] 81 29 ( 47 ) 84 65 ] 12 58 )}}
    in \texttt{( 58 [ 45 [ ( 21 18 ) 91 72 ] 81 29 ( 47 ) 84 65 ] 19 12 34 15 )}:
    applichiamo la regola 1 trasformando \texttt{( a b c d e f )} in \texttt{( f d b e c a )},
    dove a = 15, b = 19, c = 34, d = [ 45 ... 65], e = 12, f = 58.
  \item Il secondo passo trasforma
    \texttt{( 58 \underline{[ 45 [ ( 21 18 ) 91 72 ] 81 29 ( 47 ) 84 65 ]} 19 12 34 15 )}
    in \texttt{( 58 [ 65 84 ( 47 ) 29 81 [ ( 21 18 ) 91 72 ] 45 ] 19 12 34 15 )}:
    applichiamo la regola 2 trasformando [a b c d e f g ] in [ g f e d c b a ],
    dove a = 45, b = [ ( 21 18 ) 91 72 ], c = 81, d = 29, e = ( 47 ), f = 84, g = 65.
  \item Il terzo passo trasforma \texttt{( 58 [ 65 84 ( 47 ) 29 81 [ \underline{( 21 18 )} 91 72 ] 45 ] 19 12 34 15 )}
    in \texttt{( 58 [ 65 84 ( 47 ) 29 81 [ ( 18 21 ) 91 72 ] 45 ] 19 12 34 15 )}:
    applichiamo la regola 1 trasformando (a b) in (b a), dove a = 21, b = 18.
\end{itemize}

Date $N$ ESO della stessa lunghezza, il tuo compito è di
dividerle in gruppi tali che, prese due qualunque ESO in uno
stesso gruppo, queste sono equivalenti, mentre non lo sono se sono
prese da due gruppi diversi. Notare che un gruppo può essere
costituito da una sola ESO nel caso che quest'ultima non sia
equivalente a nessun'altra.

\section*{Dati di input}

Il file \texttt{input.txt} è composto da $N+1$
righe. La prima riga contiene due interi $N$ e $M$
separati da uno spazio, che rappresentano rispettivamente il numero di
ESO e la lunghezza di ciascuna di esse (ossia quanti interi e
quante parentesi contiene). 

La $I$-esima delle rimanenti $N$ righe rappresenta
la $I$-esima ESO: per facilitarne la lettura dell'input, i
numeri sono sempre positivi, la parentesi tonda aperta è codificata
dall'intero \texttt{-1}, quella chiusa dall'intero \texttt{-2}, la
parentesi quadra aperta è codificata dall'intero \texttt{-3} e quella
chiusa dall'intero \texttt{-4}; in questo modo è sufficiente leggere una
ESO come una sequenza di $M$ interi separati da uno spazio, dove i numeri
negativi rappresentano le parentesi.

\section*{Dati di output}

Il file \texttt{output.txt} è composto da $G+1$
righe. La prima riga contiene il numero $G$ di gruppi
trovati. Ciascuna delle $G$ righe successive, rappresenta un
gruppo e contiene interi separati da uno spazio: il primo intero e' il
numero $E$ di elementi nel gruppo, e gli altri rappresentano
le $E$ ESO che
appartengono al gruppo, dove la $I$-esima ESO dell'input e'
rappresentata dall'intero positivo $I$ nell'output.

\section*{Assunzioni}
\begin{itemize}
  \item $1 ≤ N ≤ 1000$
  \item $1 ≤ M ≤ 1000$
  \item $1 ≤ G ≤ 1000$
  \item Una ESO contiene al massimo $1000$ interi (inclusi i negativi per le parentesi)
\end{itemize}

\section*{Esempi di input/output}
    \noindent
    \begin{tabular}{p{11cm}|p{5cm}}
    \toprule
    \textbf{File \texttt{input.txt}}
    & \textbf{File \texttt{output.txt}}
    \\
    \midrule
    \scriptsize
    \begin{verbatim}
3 25
-1 15 19 34 -3 45 -3 -1 21 18 -2 91 72 -4 81 29 -1 47 -2 84 65 -4 12 58 -2
-1 15 -3 19 -1 34 45 -2 21 18 91 72 -3 81 -4 29 -1 47 84 -2 65 12 -4 58 -2
-1 58 -3 65 84 -1 47 -2 29 81 -3 -1 18 21 -2 91 72 -4 45 -4 19 12 34 15 -2
\end{verbatim}
    &
    \scriptsize
    \begin{verbatim}
2
2 1 3
1 2
\end{verbatim}
    \\
    \bottomrule
    \end{tabular}
  
\section*{Nota/e}
\begin{itemize}
  \item Quando due ESO contengono numeri diversi, chiaramente non possono essere equivalenti.
  \item All'interno di una ESO non ci sono due numeri positivi uguali e non ci sono coppie di parentesi vuote, ovvero \texttt{()} oppure \texttt{[]}, senza alcun elemento interno.
\end{itemize}

\end{document}
