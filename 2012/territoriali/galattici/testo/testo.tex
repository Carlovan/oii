\documentclass[a4paper,11pt]{article}

\usepackage[utf8x]{inputenc}
\SetUnicodeOption{mathletters}
\SetUnicodeOption{autogenerated}

\usepackage[italian]{babel}
\usepackage{booktabs}
\usepackage{mathpazo}
\usepackage{graphicx}
\usepackage[left=2cm, right=2cm, bottom=3cm]{geometry}
\frenchspacing

\begin{document}
\noindent {\Large Selezioni territoriali 2011}
\vspace{0.5cm}

\noindent {\Huge Eserciti galattici (\texttt{galattici})}


\vspace{0.5cm}
\noindent {\Large Difficoltà D = 2.}

\section*{Descrizione del problema}

L'esercito della Signoria è riuscito a costruire un'arma
segreta: il temibile Sarcofago Nero. Esso legge una parola
segreta $S$ costituita da lettere minuscole dell'alfabeto: a,
b, c, ..., z (ogni lettera può comparire zero, una o più
volte).

Il Sarcofago Nero può assumere $N$ configurazioni al
suo interno, numerate da $1$ a $N$. La parola
segreta $S$ viene accettata se raggiunge la configurazione
finale (avente numero $N$) a partire dalla configurazione
iniziale (avente numero $1$) dopo aver letto \emph{tutte} le lettere
in $S$ una alla volta. Per ogni configurazione $I$
del Sarcofago Nero, la tripletta $(I,J,c)$ indica che la
lettera $c$ lo fa transitare dalla
configurazione $I$ alla configurazione $J$.

L'esercito rivale ha carpito una parola segreta $S$, ma non
sa se è quella del Sarcofago Nero. Il tuo compito è
quello di trovare la configurazione interna $Q$ che esso
raggiunge, dopo aver letto $S$, a partire dalla
configurazione iniziale.


\section*{Dati di input}

Il file \texttt{input.txt} è composto da $M+2$
righe.  La prima riga contiene tre interi positivi separati da uno
spazio, che rappresentano il numero $M$ delle triplette, il
numero $N$ di configurazioni e il numero $K$ di
lettere nella sequenza $S$.  La seconda riga
contiene $K$ lettere separate da uno spazio, le quali
formano la sequenza $S$.  Ciascuna delle
rimanenti $M$ righe contiene due interi
positivi $I$ e $J$ e una lettera $c$,
separati da una spazio, che rappresentano la
tripletta $(I,J,c)$ per la transizione del Sarcofago Nero.


\section*{Dati di output}

Il file \texttt{output.txt} è composto da una sola riga
contenente il numero $Q$ della configurazione raggiunta dal
Sarcofago Nero a partire dalla sua configurazione iniziale (avente
numero 1), dopo aver letto tutta la sequenza $S$.

\section*{Assunzioni}
\begin{itemize}
  \item $2 ≤ M ≤ 100$.
  \item $2 ≤ N ≤ 100$.
  \item $2 ≤ K ≤ 10$.
  \item $1 ≤ Q ≤ N$.
\end{itemize}

\section*{Esempi di input/output}
    \noindent
    \begin{tabular}{p{11cm}|p{5cm}}
    \toprule
    \textbf{File \texttt{input.txt}}
    & \textbf{File \texttt{output.txt}}
    \\
    \midrule
    \scriptsize
    \begin{verbatim}
5 3 6
a a a b a b
1 3 a 
1 2 b
2 1 a
3 2 b
3 3 a
\end{verbatim}
    &
    \scriptsize
    \begin{verbatim}
2
\end{verbatim}
    \\
    \bottomrule
    \end{tabular}

\section*{Nota/e}
\begin{itemize}
  \item Se il Sarcofago Nero si trova nella configurazione $I$ e
    arriva la lettera $c$, viene garantita l'esistenza della
    tripletta $(I,J,c)$ per una qualche configurazione $J$.
  \item Per completare la storia, chiaramente $S$ è una
    parola segreta se $Q=N$, altrimenti non lo è. Ai fini della
    risoluzione corretta dell'esercizio, è sufficiente restituire il
    valore di $Q$.
\end{itemize}

\end{document}
