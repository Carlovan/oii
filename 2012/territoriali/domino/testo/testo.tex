\documentclass[a4paper,11pt]{article}

\usepackage[utf8x]{inputenc}
\SetUnicodeOption{mathletters}
\SetUnicodeOption{autogenerated}

\usepackage[italian]{babel}
\usepackage{booktabs}
\usepackage{mathpazo}
\usepackage{graphicx}
\usepackage[left=2cm, right=2cm, bottom=3cm]{geometry}
\frenchspacing

\begin{document}
\noindent {\Large Selezioni territoriali 2011}
\vspace{0.5cm}

\noindent {\Huge Domino massimale (\texttt{domino})}

\vspace{0.5cm}
\noindent {\Large Difficoltà D = 2.}

\section*{Descrizione del problema}

Sono date $N$ tessere di domino, dove ogni tessera
contiene due numeri compresi tra $0$ e $6$ in corrispondenza delle sue due
estremità. Le tessere possono essere ruotate e la regola impone
che due tessere possono essere concatenate se le loro estremità
in contatto hanno inciso lo stesso numero. Aiuta a trovare il maggior
numero di tessere che si possono concatenare a formare un'unica
catena: non è detto che si riescano sempre a usare tutte le
tessere; inoltre, possono esserci due o più tessere uguali a
meno di rotazioni.


\section*{Dati di input}

Il file \texttt{input.txt} è composto da $N+1$
righe.  La prima riga contiene l'intero positivo $N$, il
numero delle tessere a disposizione. Ciascuna delle
successive $N$ righe contiene due interi positivi (compresi tra $0$ e $6$) separati da
una spazio, che rappresentano i numeri incisi sulle estremità
delle tessere.


\section*{Dati di output}

Il file \texttt{output.txt} è composto da una sola riga
contenente il massimo numero di tessere che possono essere concatenate
con le regole del domino.

\section*{Assunzioni}

\begin{itemize}
  \item $2 ≤ N ≤ 10$.
\end{itemize}

\section*{Esempi di input/output}
    \noindent
    \begin{tabular}{p{11cm}|p{5cm}}
    \toprule
    \textbf{File \texttt{input.txt}}
    & \textbf{File \texttt{output.txt}}
    \\
    \midrule
    \scriptsize
    \begin{verbatim}
6
3 0
4 0
2 6
4 4
0 1
1 0
\end{verbatim}
    &
    \scriptsize
    \begin{verbatim}
5
\end{verbatim}
    \\
    \bottomrule
    \end{tabular}

\section*{Nota/e}

\begin{itemize}
  \item In generale, più configurazioni possono soddisfare i requisti del problema: è sufficiente fornire la lunghezza massima.
\end{itemize}

\end{document}
