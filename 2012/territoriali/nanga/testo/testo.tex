\documentclass[a4paper,11pt]{article}

\usepackage[utf8x]{inputenc}
\SetUnicodeOption{mathletters}
\SetUnicodeOption{autogenerated}

\usepackage[italian]{babel}
\usepackage{booktabs}
\usepackage{mathpazo}
\usepackage{graphicx}
\usepackage[left=2cm, right=2cm, bottom=3cm]{geometry}
\frenchspacing

\begin{document}
\noindent {\Large Selezioni territoriali 2011}
\vspace{0.5cm}

\noindent {\Huge Nanga Parbat (\texttt{nanga})}


\vspace{0.5cm}
\noindent {\Large Difficoltà D = 1.}

\section*{Descrizione del problema}
   
Durante la lunga scalata delle cime attorno al Nanga Parbat,
Reinhold Messner riesce a trasmettere al campo base, a intervalli
regolari, solo il dislivello percorso rispetto all'ultima
trasmissione. Se invia un numero positivo $P$, allora
è salito di $P$ metri rispetto alla precedente
trasmissione; se invia un numero negativo
$-P$, allora è sceso di $P$ metri rispetto
alla precedente trasmissione; se infine invia $P=0$, non ha cambiato
altitudine. Messner parte dal campo base a $5000$ metri.

I suoi collaboratori al campo base ricevono tali rilevamenti: aiutali
a identificare l'altitudine che risulta più frequentemente
rilevata in questo modo.


\section*{Dati di input}
  
Il file \texttt{input.txt} è composto da $N+1$
righe.  La prima riga contiene l'intero positivo $N$, il
numero dei rilevamenti trasmessi da Messner. Ciascuna delle successive
$N$ righe contiene un intero che rappresenta il dislivello
percorso rispetto alla precedente trasmissione. 


\section*{Dati di output}
  
Il file \texttt{output.txt} è composto da una sola riga
contenente l'altitudine che risulta più frequentemente
rilevata in questo modo dal campo base.

\section*{Assunzioni}

\begin{itemize}
  \item  $2 ≤ N ≤ 1000$.
  \item  $-100 ≤ P ≤ 100$.
\end{itemize}

\section*{Esempi di input/output}
    \noindent
    \begin{tabular}{p{11cm}|p{5cm}}
    \toprule
    \textbf{File \texttt{input.txt}}
    & \textbf{File \texttt{output.txt}}
    \\
    \midrule
    \scriptsize
    \begin{verbatim}
8
3
-1
6
-7
1
4
0
-4
\end{verbatim}
    &
    \scriptsize
    \begin{verbatim}
5002
\end{verbatim}
    \\
    \bottomrule
    \end{tabular}

\section*{Nota/e}

\begin{itemize}
  \item L'altitudine iniziale viene rilevata a fini della risposta.
  \item Viene garantito nei dati di input che l'altitudine più
    frequentemente rilevata è unica.
\end{itemize}

\end{document}
