
\documentclass[a4paper,11pt]{article}

\usepackage[utf8x]{inputenc}
\SetUnicodeOption{mathletters}
\SetUnicodeOption{autogenerated}

\usepackage[italian]{babel}
\usepackage{booktabs}
\usepackage{mathpazo}
\usepackage{graphicx}
\usepackage[left=2cm, right=2cm, bottom=3cm]{geometry}
\frenchspacing

\begin{document}
\noindent {\Large Olimpiadi di Informatica: selezioni nazionali 2011}
\vspace{0.5cm}

\noindent {\Huge Fuga dagli inseguitori (\texttt{fuga})}


\section*{Descrizione del problema}
  
L'incredibile Hulk sta fuggendo dai suoi inseguitori che vogliono
catturarlo per studiare la mutazione genetica che l'ha reso
così forte, veloce e verde.  Possiede una mappa con le vie di fuga,
dove gli snodi (cioè i punti da cui si dipartono zero o più
strade) sono numerati da 1 a $N$ e il tratto di strada che
collega direttamente gli snodi $I$ e $J$ è indicato con
$(I,J)$ e può essere percorso in entrambe le
direzioni. Inoltre, non esistono due o più tratti di strada che
colleghino direttamente la stessa coppia di snodi. I tratti occupati
dagli inseguitori sono indicati in rosso e quelli liberi in verde.

Hulk vuole trovare un percorso circolare libero per la sua fuga: in
altre parole, vuole essere sicuro di poter girare circolarmente, e a
velocità spedita, attraverso gli snodi (non necessariamente
tutti).  In particolare, ha bisogno di individuare gli
snodi $I_{1}$, $I_{2}$,
$I_{3}$, …,
$I_{K}$ (dove $K ≥ 3$), distinti tra
loro, che sono collegati da un cammino, ossia da una sequenza di
tratti tutti in verde
$(I_{1}$,$I_{2}$$)$,
$(I_{2}$,$I_{3}$$)$,
…,
$(I_{K-1}$,$I_{K}$$)$,
$(I_{K}$,$I_{1}$$)$ (notare
la circolarità).

Il tuo compito è di aiutare Hulk a individuare un insieme di
snodi che dia luogo a circolarità secondo quanto definito
sopra.


\section*{Dati di input}
  
Il file \texttt{input.txt} è composto da $M+1$
righe: sulla prima riga si trovano gli interi $N$ e
$M$ separati da uno spazio, dove $N$ è il
numero di snodi e $M$ è il numero di tratti che
collegano gli snodi.

Ciascuna delle successive $M$ righe contiene tre interi
$I$, $J$ e $C$ separati da uno spazio,
dove $1 ≤ I, J ≤ N$ e $0 ≤ C ≤ 1$,
per indicare che gli snodi
$I$ e $J$ sono collegati dal tratto $(I,J)$
di colore rosso ($C=0$) o verde ($C=1$).


\section*{Dati di output}
  
Il file \texttt{output.txt} è composto da due righe.  La
prima riga contiene un intero $K$ che indica quanti snodi
sono coinvolti nella circolarità individuata. La seconda riga
contiene $K$ interi distinti $I_{1}$,
$I_{2}$,$I_{3}$,…,
$I_{K}$ separati da uno spazio, ossia quali sono gli
snodi coinvolti: essi risultano collegati da tratti in verde
$(I_{1}$,$I_{2}$$)$,
$(I_{2}$,$I_{3}$$)$,
…,
$(I_{K-1}$,$I_{K}$$)$,
$(I_{K}$,$I_{1}$$)$ .

  \section*{Assunzioni}
  \begin{itemize}
  
    \item $3 ≤ N ≤ 100 000$
    \item $N < M ≤ 200 000$
    \item $3 ≤ K ≤ N$
    \item Viene garantito che esiste sempre almeno una circolarità.
  \end{itemize}

\section*{Esempi di input/output}

  
    \noindent
    \begin{tabular}{p{11cm}|p{5cm}}
    \toprule
    \textbf{File \texttt{input.txt}}
    & \textbf{File \texttt{output.txt}}
    \\
    \midrule
    \scriptsize
    \begin{verbatim}
5 8
5 4 1
1 2 1
1 3 1
3 2 1
1 5 0
1 4 1
3 4 0
2 5 1
\end{verbatim}
    &
    \scriptsize
    \begin{verbatim}
3
2 3 1
\end{verbatim}
    \\
    \bottomrule
    \end{tabular}
  
\section*{Nota/e}
\begin{itemize}
  
    \item  Non tutti gli snodi hanno necessariamente una o più strade che si
  dipartono da loro (potrebbero esserci snodi completamente isolati).
    \item  Non tutte le coppie di snodi sono necessariamente collegate tra
  di loro mediante un tratto o una sequenza di tratti.
    \item  Per un dato \texttt{input.txt} ci possono essere più
  risposte corrette e sono tutte valide ai fini della gara: è necessario
  specificarne una (e una sola) in \texttt{output.txt}.
\end{itemize}



\end{document}
