
\documentclass[a4paper,11pt]{article}

\usepackage[utf8x]{inputenc}
\SetUnicodeOption{mathletters}
\SetUnicodeOption{autogenerated}

\usepackage[italian]{babel}
\usepackage{booktabs}
\usepackage{mathpazo}
\usepackage{graphicx}
\usepackage[left=2cm, right=2cm, bottom=3cm]{geometry}
\frenchspacing

\begin{document}
\noindent {\Large Olimpiadi di Informatica: selezioni nazionali 2011}
\vspace{0.5cm}

\noindent {\Huge Salti spettacolari (\texttt{salti})}


\section*{Descrizione del problema}
   Quando il Dr. Bruce Banner si trasforma nell'incredibile Hulk,
acquista sempre più forza con l'andare del tempo. Al
tempo $t=0$ riesce a saltare un solo metro, al
tempo $t=1$ ne salta due, al tempo $t=2$ ne salta
quattro e così via: in generale, al tempo $t ≥ 0$,
riesce a saltare $2^{t}$ metri. Tuttavia l'incredibile Hulk
può saltare sempre e solo nella stessa direzione: dunque ad ogni istante $t$
può decidere se saltare in avanti alla distanza permessagli in quel momento
oppure stare fermo.

Hulk deve percorrere una certa distanza $D > 0$, espressa
in metri, e vuole effettuare il minor numero di salti. Per esempio,
per $D=9$, Hulk salta due volte (effettua un salto da 1 e uno
da 8); per $D=7$, Hulk salta tre volte (un salto da 1, uno da
2 e uno da 4); per $D=16$, Hulk effettua il solo salto da 16.

Aiuta Hulk a calcolare quale è il minimo numero di salti che
deve effettuare per coprire la distanza $D$.


\section*{Dati di input}
  
Il file \texttt{input.txt} è composto da una sola riga
  contenente un intero positivo $D$, che rappresenta la
  distanza da percorrere. 


\section*{Dati di output}
  
Il file \texttt{output.txt} è composto da una sola riga che
contiene il numero di salti che Hulk deve effettuare per coprire la
distanza $D$.

  \section*{Assunzioni}
  \begin{itemize}
  
    \item $1 ≤ D ≤ 2^{30}$
  \end{itemize}

\section*{Esempi di input/output}

  
    \noindent
    \begin{tabular}{p{11cm}|p{5cm}}
    \toprule
    \textbf{File \texttt{input.txt}}
    & \textbf{File \texttt{output.txt}}
    \\
    \midrule
    \scriptsize
    \begin{verbatim}
9
\end{verbatim}
    &
    \scriptsize
    \begin{verbatim}
2
\end{verbatim}
    \\
    \bottomrule
    \end{tabular}
  
    \noindent
    \begin{tabular}{p{11cm}|p{5cm}}
    \toprule
    \textbf{File \texttt{input.txt}}
    & \textbf{File \texttt{output.txt}}
    \\
    \midrule
    \scriptsize
    \begin{verbatim}
7
\end{verbatim}
    &
    \scriptsize
    \begin{verbatim}
3
\end{verbatim}
    \\
    \bottomrule
    \end{tabular}
  
    \noindent
    \begin{tabular}{p{11cm}|p{5cm}}
    \toprule
    \textbf{File \texttt{input.txt}}
    & \textbf{File \texttt{output.txt}}
    \\
    \midrule
    \scriptsize
    \begin{verbatim}
16
\end{verbatim}
    &
    \scriptsize
    \begin{verbatim}
1
\end{verbatim}
    \\
    \bottomrule
    \end{tabular}
  
\section*{Nota/e}
\begin{itemize}
  
    \item 
Per ogni input, esiste una sola risposta da fornire in output.

\end{itemize}



\end{document}
