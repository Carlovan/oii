
\documentclass[a4paper,11pt]{article}

\usepackage[utf8x]{inputenc}
\SetUnicodeOption{mathletters}
\SetUnicodeOption{autogenerated}

\usepackage[italian]{babel}
\usepackage{booktabs}
\usepackage{mathpazo}
\usepackage{graphicx}
\usepackage[left=2cm, right=2cm, bottom=3cm]{geometry}
\frenchspacing

\begin{document}
\noindent {\Large Selezioni territoriali 2010}
\vspace{0.5cm}

\noindent {\Huge Sbarramento tattico (\texttt{sbarramento})}


\vspace{0.5cm}
\noindent {\Large Difficoltà D = 2.}

\section*{Descrizione del problema}
  
L'esercito di Orchi dell'Oscuro Signore degli Anelli marcia a ranghi
serrati verso il Fosso di Helm.  Per contrastarne la marcia, Re
Theoden decide di richiamare tutte le sue $N$ armate per
creare uno sbarramento unico, con le seguenti regole.

\begin{itemize}
  
    \item 
Campo di battaglia: è rappresentato da una tabella di
dimensione $NxN$, le cui righe e colonne sono numerate da 1 a
$N$.

    \item 
Posizione: ognuna delle $N$ armate occupa una posizione
distinta $[i,j]$
nella tabella, all'incrocio tra la riga $i$ e la
colonna $j$.

    \item 
Movimento: permette di passare dalla posizione corrente
$[i,j]$ a una vicina con un giorno di marcia: nord
$[i-1,j]$ (se $i > 1$), sud $[i+1,j]$
(se $i < N$), est $[i,j+1]$ (se $j <
N$) e ovest $[i,j-1]$ (se $j > 1$). 
Una sola armata alla volta si sposta con un movimento.

    \item 
Sbarramento: si crea ponendo tutte le armate su un'unica
riga $R$ della tabella, attraverso una serie di
movimenti. 

\end{itemize}


Theoden vuole calcolare il numero minimo di movimenti necessari per
spostare tutte le armate in un unico sbarramento sulla riga
$R$.  Aiutate Theoden a calcolare tale numero minimo.


\section*{Dati di input}
  
Il file \texttt{input.txt} è composto da $N+1$
righe.  La prima riga contiene due interi positivi $N$
e $R$, separati da uno spazio: il numero $N$ di
righe e di colonne nella tabella (nonché il numero di armate) e
l'indice $R$ della riga su cui far convergere lo sbarramento
delle armate.  Ciascuna delle successive $N$ righe contiene
una coppia di interi $i$ e $j$, separati da uno
spazio, a indicare che un'armata è presente nella
posizione $[i,j]$ della tabella.


\section*{Dati di output}
  
Il file \texttt{output.txt} è composto da 
una sola riga contenente un intero non negativo, il minimo numero di movimenti per posizionare
tutte le armate sulla riga $R$ della tabella, in posizioni
distinte all'interno di tale riga.

  \section*{Assunzioni}
  \begin{itemize}
  
    \item  2 ≤ $N$ ≤ 500.
    \item  Durante un movimento, due o più armate non possono mai occupare la stessa
  posizione intermedia.
  \end{itemize}

\section*{Esempi di input/output}

  
    \noindent
    \begin{tabular}{p{11cm}|p{5cm}}
    \toprule
    \textbf{File \texttt{input.txt}}
    & \textbf{File \texttt{output.txt}}
    \\
    \midrule
    \scriptsize
    \begin{verbatim}
8 3
5 5
1 6
2 2
6 5
3 2
7 1
1 2
8 1
\end{verbatim}
    &
    \scriptsize
    \begin{verbatim}
31
\end{verbatim}
    \\
    \bottomrule
    \end{tabular}
  
    \noindent
    \begin{tabular}{p{11cm}|p{5cm}}
    \toprule
    \textbf{File \texttt{input.txt}}
    & \textbf{File \texttt{output.txt}}
    \\
    \midrule
    \scriptsize
    \begin{verbatim}
8 5
5 7
5 2
5 3
5 6
5 1
5 8
5 5
5 4
\end{verbatim}
    &
    \scriptsize
    \begin{verbatim}
0
\end{verbatim}
    \\
    \bottomrule
    \end{tabular}
  
\section*{Nota/e}
\begin{itemize}
  
    \item Un programma che restituisce sempre lo stesso valore,
indipendentemente dai dati in \texttt{input.txt}, non totalizza
alcun punteggio.
\end{itemize}



\end{document}
