
\documentclass[a4paper,11pt]{article}

\usepackage[utf8x]{inputenc}
\SetUnicodeOption{mathletters}
\SetUnicodeOption{autogenerated}

\usepackage[italian]{babel}
\usepackage{booktabs}
\usepackage{mathpazo}
\usepackage{graphicx}
\usepackage[left=2cm, right=2cm, bottom=3cm]{geometry}
\frenchspacing

\begin{document}
\noindent {\Large Selezioni territoriali 2010}
\vspace{0.5cm}

\noindent {\Huge Sequenza per tamburello (\texttt{tamburello})}


\vspace{0.5cm}
\noindent {\Large Difficoltà D = 1.}

\section*{Descrizione del problema}
   
Marco ha trovato alcune antiche sequenze in un manoscritto. Ogni
sequenza è composta da $N$ pallini pieni o vuoti e
rappresenta un brano da suonare al tamburello in $N$ istanti
consecutivi di tempo: all'$i$-esimo istante, il tamburello
viene percosso se l'$i$-esimo pallino è pieno e, invece,
non viene percosso se tale pallino è vuoto ($1 <= i <= N$).

Marco vuole capire se una data sequenza è periodica: in tal
caso, vuole estrarne il \textbf{periodo}, ossia il più piccolo segmento
iniziale che si ripete nel resto della sequenza. In altre parole,
se $P$ è la sequenza di pallini pieni e vuoti che rappresenta il periodo,
allora la sequenza in input è periodica se può essere ottenuta
concatenando $P$ per due o più volte e tale $P$
deve essere di lunghezza minima.

Per esempio, rappresentando con \texttt{1} ogni pallino pieno
e con \texttt{0} ogni pallino vuoto, la sequenza periodica
\texttt{101010101010} ha \texttt{10} come periodo e la
sequenza \texttt{1010010100010100101000}
ha \texttt{10100101000} come periodo. Invece, la
sequenza \texttt{11011011} non è periodica. Aiutate
Marco in questo compito, in modo che possa imparare a suonare
velocemente tali brani per tamburello.


\section*{Dati di input}
  
Il file \texttt{input.txt} è composto da due righe.  La prima
riga contiene un intero positivo $N$, che indica il numero di pallini
nella sequenza. La seconda riga contiene una sequenza di
interi \texttt{0} e \texttt{1}, separati da uno spazio,
dove \texttt{1} rappresenta un pallino pieno e \texttt{0}
un pallino vuoto.


\section*{Dati di output}
  
Il file \texttt{output.txt} è composto da una sola riga
contenente l'intero \texttt{2} se la sequenza in input non è
periodica. Altrimenti, se è periodica, la riga contiene la sequenza
di \texttt{0} e \texttt{1}, separati da uno spazio, che
rappresenta il periodo $P$ della sequenza fornita in input.

  \section*{Assunzioni}
  \begin{itemize}
  
    \item  2 ≤ $N$ ≤ 100000.
  \end{itemize}

\section*{Esempi di input/output}

  
    \noindent
    \begin{tabular}{p{11cm}|p{5cm}}
    \toprule
    \textbf{File \texttt{input.txt}}
    & \textbf{File \texttt{output.txt}}
    \\
    \midrule
    \scriptsize
    \begin{verbatim}
12
1 0 1 0 1 0 1 0 1 0 1 0
\end{verbatim}
    &
    \scriptsize
    \begin{verbatim}
1 0
\end{verbatim}
    \\
    \bottomrule
    \end{tabular}
  
    \noindent
    \begin{tabular}{p{11cm}|p{5cm}}
    \toprule
    \textbf{File \texttt{input.txt}}
    & \textbf{File \texttt{output.txt}}
    \\
    \midrule
    \scriptsize
    \begin{verbatim}
22
1 0 1 0 0 1 0 1 0 0 0 1 0 1 0 0 1 0 1 0 0 0
\end{verbatim}
    &
    \scriptsize
    \begin{verbatim}
1 0 1 0 0 1 0 1 0 0 0
\end{verbatim}
    \\
    \bottomrule
    \end{tabular}
  
    \noindent
    \begin{tabular}{p{11cm}|p{5cm}}
    \toprule
    \textbf{File \texttt{input.txt}}
    & \textbf{File \texttt{output.txt}}
    \\
    \midrule
    \scriptsize
    \begin{verbatim}
8
1 1 0 1 1 0 1 1
\end{verbatim}
    &
    \scriptsize
    \begin{verbatim}
2
\end{verbatim}
    \\
    \bottomrule
    \end{tabular}
  
\section*{Nota/e}
\begin{itemize}
  
    \item Un programma che restituisce sempre lo stesso valore,
indipendentemente dai dati in \texttt{input.txt}, non totalizza
alcun punteggio.
\end{itemize}



\end{document}
